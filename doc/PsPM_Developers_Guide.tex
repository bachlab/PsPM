%% LyX 2.3.6 created this file.  For more info, see http://www.lyx.org/.
%% Do not edit unless you really know what you are doing.
\documentclass[english]{article}
\usepackage[T1]{fontenc}
\usepackage[latin9]{inputenc}
\usepackage{color}
\usepackage[british,UKenglish]{babel}
\usepackage{array}
\usepackage{longtable}
\usepackage{bbding}
\usepackage{rotating}
\usepackage{setspace}
\usepackage[unicode=true]
 {hyperref}

\makeatletter

%%%%%%%%%%%%%%%%%%%%%%%%%%%%%% LyX specific LaTeX commands.
%% Because html converters don't know tabularnewline
\providecommand{\tabularnewline}{\\}

%%%%%%%%%%%%%%%%%%%%%%%%%%%%%% User specified LaTeX commands.
\definecolor{colKeys}{rgb}{0,0,1}
\definecolor{colIdentifier}{rgb}{0,0,0}
\definecolor{colComments}{rgb}{0.53, 0.66, 0.42}
\definecolor{colString}{rgb}{0.87, 0.36, 0.51}
\definecolor{barColor}{rgb}{0.43, 0.5, 0.5}
% Added by lyx2lyx
\renewcommand{\textendash}{--}
\renewcommand{\textemdash}{---}
\usepackage[cmintegrals,cmbraces]{newtxmath}
\usepackage{ebgaramond-maths}
\usepackage[T1]{fontenc}
\makeatother

\usepackage{listings}
\lstset{language=Matlab,
float=hbp,
basicstyle={\footnotesize\ttfamily},
identifierstyle={\color{colIdentifier}},
keywordstyle={\color{colKeys}},
stringstyle={\color{colString}},
commentstyle={\itshape\color{colComments}},
columns=fixed,
tabsize=2,
extendedchars=true,
showspaces=false,
showstringspaces=false,
captionpos=t,
backgroundcolor={\color{white}},
framexleftmargin=1pt,
frame=l}
\renewcommand{\lstlistingname}{Listing}

\begin{document}
\title{PsPM: Psychophysiological Modelling}
\title{Developer's Guide}
\maketitle
\begin{center}
Version 5.1.1
\par\end{center}

\medskip{}

\begin{center}
by the PsPM team\footnote{\noindent If you have comments on or error corrections to this documentation,
please send them to the PsPM team or post them on: \href{http://bachlab.org/pspm}{bachlab.org/pspm}}{\Large{}:}{\Large\par}
\par\end{center}

\begin{center}
Dominik R Bach, Giuseppe Castegnetti, Laure Ciernik, Samuel Gerster,
Saurabh Khemka, Christoph Korn, Samuel Maxwell, Tobias Moser, Philipp
C Paulus, Ivan Rojkov, Matthias Staib, Eshref Yozdemir, Dadi Zhao
and collaborators\\
\par\end{center}
\clearpage
\newpage{}
\tableofcontents{}
\newpage{}

\section{General}

\subsection{Data files: General structure }

In PsPM the data is saved in mat-files. Each file contains two variables: 
\begin{itemize}
\item \texttt{infos} A struct variable with general infos 
\item \texttt{data} A cell array with a cell for each channel. 
\end{itemize}
The cells contain a struct with channel specific infos and data. The
structs have the mandatory fields: 
\begin{itemize}
\item \texttt{infos.duration} (in seconds)
\item \texttt{data\{n\}.header} 

\begin{itemize}
\item \texttt{data\{n\}.header.chantype} (as defined in the settings) 
\item \texttt{data\{n\}.header.sr} (sample rate in 1/second, or timestamp
units in seconds) 
\item \texttt{data\{n\}.header.units} (data units, or \textquoteleft events\textquoteright ) 
\item \texttt{data\{n\}.data} (actual data) 
\end{itemize}
\end{itemize}
Additionally, a typical file contains the optional infos: 
\begin{itemize}
\item \texttt{infos.sourcefile }
\item \texttt{infos.importfile }
\item \texttt{infos.importdate }
\item \texttt{infos.sourcetype }
\item \texttt{infos.recdate }
\item \texttt{infos.rectime }
\end{itemize}
Some data manipulation functions (in particular, pspm\_trim) update
infos to record some file history. 

\subsection{How to add a new import data type}

\subsubsection{Add function}

\begin{singlespace}
\noindent Function name: pspm\_get\_xxx (where xxx is the data type
name). \\
\\
Format: \\

\noindent \texttt{{[}sts, import, sourceinfo{]} = pspm\_get\_xxx(datafile,
import)}~\\
\texttt{}~\\
The function needs to take an import job and add, for each job, fields 
\end{singlespace}
\begin{itemize}
\item .data - the actual data for this channel (column vector) 
\item .sr - the sample rate for this channel (only if .autosr enabled in
pspm\_init) 
\end{itemize}
\noindent optional fields 
\begin{itemize}
\item .marker - for marker channels (timestamps or continuous, see pspm\_get\_marker) 
\item .markerinfo -- optional, see pspm\_get\_marker 
\item .minfreq - minimum frequency for pulse channels 
\item .units - if data units are defined by the recording software 
\item sts: -1 if import is unsuccessful 
\end{itemize}
\noindent sourceinfo: contains information on the source file, with
field 
\begin{itemize}
\item .chan - a cell of string descriptions of the imported source channels,
e. g. names, or numbers any optional fields that will be added to
infos.source (e. g. recording date \& time, and others)\texttt{ }
\end{itemize}
Notes for multiple blocks: file formats that support multiple block
storage within one file can return cell arrays import\{1:blkno\} and
sourceinfo\{1:blkno\}; PsPM will save individual files for each block,
with a filename 'pspm\_fn\_blk0x.mat'.

\subsubsection{Add information to settings}

The file pspm\_init contains a block that defines possible import
data types. Add a new field here\\

\noindent \texttt{}
\begin{lstlisting}
% Description of data type 
% --------------------------------------------- 
defaults.import.datatypes(1) = ... 
struct('short', 'xxx', ... % short name for internal purposes 
'long', 'Datatype description', ... % long name for GUI 
'ext', '*', ... % data file extension 
'funct', @pspm_get_xxx, ... % import function 
'chantypes', {{defaults.chantypes.type}}, ... % allowed channel types 
'chandescription', 'channel', ... % description of channels for GUI 
'multioption', 1, ... % import of multiple channels for GUI 
'searchoption', 1, ... % allow channel name search for GUI 
'automarker', 0, ... % marker not stored in separate channel 
'autosr', 1); % sample rate automatically assigned
\end{lstlisting}

Good to know: 
\begin{itemize}
\item the \textquotedblleft long\textquotedblright{} definition is used
in the GUI -- make sure it\textquoteright s readable 
\item if no event channels can be imported, change .chantypes 
\item if channels have searchable names in the import file, set .searchoption
= 1
\item if no channel number needs to be assigned for the marker channel,
set .automarker = 1
\item if sample rate is contained in import file and determined during import,
set .autosr = 1
\item if you need external functions -- put them into a folder in the \textquoteleft import\textquoteright{}
subdirectory and add/remove this path within the pspm\_get\_xxx function
\end{itemize}

\subsection{How to add a new channel type}

\subsubsection{Add function}

Function name: pspm\_get\_xxx (where xxx is the channel type)\\
\\
Format:\\
\\
\texttt{{[}sts, data{]} = pspm\_get\_channeltype(import)}~\\
\texttt{}~\\
data: data cell of structure readable by pspm\_load\_data \\
\\
Good to know: for event channels, use the function pspm\_get\_events
to convert various event formats into time stamps (see pspm\_get\_marker
or pspm\_get\_hb as an example) 

\subsubsection{Add information to settings}

Add information on the new channel type and import function to\\
\\
\texttt{}
\begin{lstlisting}
defaults.chantypes(k).type = 'xxx';% channel type name 
defaults.chantypes(k).import = @pspm_get_xxx; % conversion function
defaults.chantypes(k).data = 'xxx'; % wave or events 
\end{lstlisting}


\subsection{How to add a new GLM type}

\subsubsection{Add information to settings (Example SCR)}

\texttt{}
\begin{lstlisting}
defaults.glm(1) = ... 
struct('modality', 'scr', ... % modality name 
'cbf', struct('fhandle', @pspm_bf_scrf, 'args', 1), ... 
% default basis function/set
'filter', struct('lpfreq', 5, 'lporder', 1, ... 
'hpfreq', 0.05, 'hporder', 1, 'down', 10, 'direction', 'uni')); 
% default filter settings 
\end{lstlisting}


\subsubsection{Add default basis function}

Function name: pspm\_bf\_xxx \\
\\
Function arguments: vector of arguments, first element is time resolution,
further arguments as defined in defaults.glm(n).cbf.args 

\subsection{Warning IDs in PsPM}

\subsubsection{General}
\begin{itemize}
\item invalid\_input
\item invalid\_channeltype
\item nonexistent\_file
\item channel\_not\_contained\_in\_file
\item obsolete\_function
\item not\_allowed\_channeltype
\item invalid\_data\_structure
\item no\_matching\_channels
\item unknown\_action
\item missing\_data
\item out\_of\_range
\end{itemize}

\subsubsection{Function specific}

pspm\_load1
\begin{itemize}
\item not\_saving\_data
\end{itemize}
pspm\_interpolate
\begin{itemize}
\item option\_disabled
\end{itemize}
pspm\_trim
\begin{itemize}
\item marker\_out\_of\_range
\end{itemize}
pspm\_find\_channel
\begin{itemize}
\item multiple\_matching\_channels
\end{itemize}
pspm\_find\_sounds
\begin{itemize}
\item no\_marker\_chan
\item no\_sound\_chan
\end{itemize}
pspm\_get\_scr
\begin{itemize}
\item no\_conversion\_constant
\end{itemize}
pspm\_pp
\begin{itemize}
\item invalid\_freq
\end{itemize}
pspm\_prepdata
\begin{itemize}
\item no\_low\_pass\_filtering
\item downsampling\_failed
\item nonint\_sr 
\end{itemize}
pspm\_get\_timing
\begin{itemize}
\item invalid\_vector\_size
\item event\_names\_dont\_match
\item no\_numeric\_vector
\item no\_integers 
\end{itemize}
pspm\_down
\begin{itemize}
\item rate\_below\_minimum
\end{itemize}

\section{List of data formats}

\subsection{Supported Channel types}

\begin{tabular}{|>{\centering}p{3.6cm}||c|c|c|c|c|c|c|c|c|c|c|c|}
\hline 
Data format & \begin{turn}{90}
SCR
\end{turn} & \begin{turn}{90}
ECG
\end{turn} & \begin{turn}{90}
Heart Rate
\end{turn} & \begin{turn}{90}
Heart Beat
\end{turn} & \begin{turn}{90}
Heart Period
\end{turn} & \begin{turn}{90}
Respiration
\end{turn} & \begin{turn}{90}
Pupil Size
\end{turn} & \begin{turn}{90}
Marker
\end{turn} & \begin{turn}{90}
Custom
\end{turn} & \begin{turn}{90}
Sound channel
\end{turn} & \begin{turn}{90}
Pulse oxymeter
\end{turn} & \begin{turn}{90}
Gaze x/y, l/r
\end{turn}\tabularnewline
\hline 
\hline 
CED Spike & \Checkmark{} & \Checkmark{} & \Checkmark{} & \Checkmark{} & \Checkmark{} & \Checkmark{} & \Checkmark{} & \Checkmark{} & \Checkmark{} & \Checkmark{} & \Checkmark{} & \Checkmark{}\tabularnewline
\hline 
Matlab & \Checkmark{} & \Checkmark{} & \Checkmark{} & \Checkmark{} & \Checkmark{} & \Checkmark{} & \Checkmark{} & \Checkmark{} & \Checkmark{} & \Checkmark{} & \Checkmark{} & \Checkmark{}\tabularnewline
\hline 
Text & \Checkmark{} & \Checkmark{} & \Checkmark{} &  & \Checkmark{} & \Checkmark{} & \Checkmark{} & \Checkmark{} & \Checkmark{} & \Checkmark{} & \Checkmark{} & \Checkmark{}\tabularnewline
\hline 
Biopach AcqKnowledge ($\leq$ v3.9.0) & \Checkmark{} & \Checkmark{} & \Checkmark{} & \Checkmark{} & \Checkmark{} & \Checkmark{} & \Checkmark{} & \Checkmark{} & \Checkmark{} & \Checkmark{} & \Checkmark{} & \Checkmark{}\tabularnewline
\hline 
Biopac AcqKnowledge (exported) & \Checkmark{} & \Checkmark{} & \Checkmark{} & \Checkmark{} & \Checkmark{} & \Checkmark{} & \Checkmark{} & \Checkmark{} & \Checkmark{} & \Checkmark{} & \Checkmark{} & \Checkmark{}\tabularnewline
\hline 
Labchart (any Version, Windows only) & \Checkmark{} & \Checkmark{} & \Checkmark{} &  & \Checkmark{} & \Checkmark{} & \Checkmark{} & \Checkmark{} & \Checkmark{} & \Checkmark{} & \Checkmark{} & \Checkmark{}\tabularnewline
\hline 
Labchart exported 

($\leq$ v7.1) & \Checkmark{} & \Checkmark{} & \Checkmark{} &  & \Checkmark{} & \Checkmark{} & \Checkmark{} & \Checkmark{} & \Checkmark{} & \Checkmark{} & \Checkmark{} & \Checkmark{}\tabularnewline
\hline 
Labchart exported 

($\geq$ v7.2) & \Checkmark{} & \Checkmark{} & \Checkmark{} &  & \Checkmark{} & \Checkmark{} & \Checkmark{} & \Checkmark{} & \Checkmark{} & \Checkmark{} & \Checkmark{} & \Checkmark{}\tabularnewline
\hline 
VarioPort & \Checkmark{} & \Checkmark{} & \Checkmark{} &  & \Checkmark{} & \Checkmark{} & \Checkmark{} & \Checkmark{} & \Checkmark{} & \Checkmark{} & \Checkmark{} & \Checkmark{}\tabularnewline
\hline 
Biograph Infiniti (exported) & \Checkmark{} &  &  & \Checkmark{} &  & \Checkmark{} &  &  &  &  &  & \tabularnewline
\hline 
Mindmedia Biotrace (exported) & \Checkmark{} & \Checkmark{} & \Checkmark{} &  & \Checkmark{} & \Checkmark{} & \Checkmark{} & \Checkmark{} & \Checkmark{} & \Checkmark{} & \Checkmark{} & \Checkmark{}\tabularnewline
\hline 
Brain Vision & \Checkmark{} & \Checkmark{} & \Checkmark{} &  & \Checkmark{} & \Checkmark{} & \Checkmark{} & \Checkmark{} & \Checkmark{} & \Checkmark{} & \Checkmark{} & \Checkmark{}\tabularnewline
\hline 
Windaq (wdq) & \Checkmark{} & \Checkmark{} & \Checkmark{} & \Checkmark{} & \Checkmark{} & \Checkmark{} & \Checkmark{} & \Checkmark{} & \Checkmark{} & \Checkmark{} &  & \Checkmark{}\tabularnewline
\hline 
Observer XT compatible & \Checkmark{} & \Checkmark{} & \Checkmark{} & \Checkmark{} & \Checkmark{} & \Checkmark{} & \Checkmark{} & \Checkmark{} & \Checkmark{} & \Checkmark{} & \Checkmark{} & \Checkmark{}\tabularnewline
\hline 
NeuroScan & \Checkmark{} & \Checkmark{} & \Checkmark{} &  & \Checkmark{} & \Checkmark{} & \Checkmark{} & \Checkmark{} & \Checkmark{} & \Checkmark{} & \Checkmark{} & \Checkmark{}\tabularnewline
\hline 
BioSemi & \Checkmark{} & \Checkmark{} & \Checkmark{} &  & \Checkmark{} & \Checkmark{} & \Checkmark{} & \Checkmark{} & \Checkmark{} & \Checkmark{} & \Checkmark{} & \Checkmark{}\tabularnewline
\hline 
Eyelink &  &  &  &  &  &  & \Checkmark{} & \Checkmark{} & \Checkmark{} &  &  & \Checkmark{}\tabularnewline
\hline 
European Data Format & \Checkmark{} & \Checkmark{} & \Checkmark{} & \Checkmark{} & \Checkmark{} & \Checkmark{} & \Checkmark{} & \Checkmark{} & \Checkmark{} & \Checkmark{} & \Checkmark{} & \Checkmark{}\tabularnewline
\hline 
Philips Scanphyslog &  & \Checkmark{} &  &  &  & \Checkmark{} &  & \Checkmark{} & \Checkmark{} &  & \Checkmark{} & \tabularnewline
\hline 
SMI &  &  &  &  &  &  & \Checkmark{} & \Checkmark{} & \Checkmark{} &  &  & \Checkmark{}\tabularnewline
\hline 
ViewPoint &  &  &  &  &  &  & \Checkmark{} & \Checkmark{} & \Checkmark{} &  &  & \Checkmark{}\tabularnewline
\hline 
\end{tabular}

\subsection{Further settings}

\begin{tabular}{|>{\centering}p{3.6cm}||c|c|>{\centering}p{2.2cm}|c|c|c|c|}
\hline 
Data format & Datatype & \begin{turn}{90}
File extension
\end{turn} & Manufacturer & \begin{turn}{90}
Import multiple channels
\end{turn} & \begin{turn}{90}
Search channel names
\end{turn} & \begin{turn}{90}
Automarker
\end{turn} & \begin{turn}{90}
Ask for sampling rate
\end{turn}\tabularnewline
\hline 
\hline 
CED Spike & spike & .smr & CED & \Checkmark{} & \Checkmark{} &  & \tabularnewline
\hline 
Matlab & mat & .mat &  & \Checkmark{} &  &  & \Checkmark{}\tabularnewline
\hline 
Text & txt & .txt &  & \Checkmark{} & \Checkmark{} &  & \tabularnewline
\hline 
Biopach AcqKnowledge ($\leq$ v3.9.0) & acq & .acq & Biopac & \Checkmark{} & \Checkmark{} &  & \tabularnewline
\hline 
Biopac AcqKnowledge (exported) & acqmat & .mat & Biopac & \Checkmark{} & \Checkmark{} &  & \tabularnewline
\hline 
Labchart (any Version, Windows only) & labchart & .adicht & ADInstruments & \Checkmark{} & \Checkmark{} & \Checkmark{} & \tabularnewline
\hline 
Labchart exported 

($\leq$ v7.1) & labchartmat\_ext & .mat & ADInstruments & \Checkmark{} & \Checkmark{} & \Checkmark{} & \Checkmark{}\tabularnewline
\hline 
Labchart exported 

($\geq$ v7.2) & labchartmat\_in & .mat & ADInstruments & \Checkmark{} &  & \Checkmark{} & \tabularnewline
\hline 
VarioPort & vario & .vpd & Becker MediTec & \Checkmark{} & \Checkmark{} & \Checkmark{} & \tabularnewline
\hline 
Biograph Infiniti (exported) & biograph & .txt & Thought Technology &  &  &  & \tabularnewline
\hline 
Mindmedia Biotrace (exported) & biotrace & .txt & MindMedia &  &  & \Checkmark{} & \tabularnewline
\hline 
Brain Vision & brainvision & .eeg & BrainProducts & \Checkmark{} & \Checkmark{} & \Checkmark{} & \tabularnewline
\hline 
Windaq (wdq) & windaq & .wdq & Dataq & \Checkmark{} &  &  & \tabularnewline
\hline 
Observer XT compatible & observer & .any & Noldus & \Checkmark{} & \Checkmark{} &  & \tabularnewline
\hline 
NeuroScan & cnt & .cnt &  & \Checkmark{} & \Checkmark{} & \Checkmark{} & \tabularnewline
\hline 
BioSemi & biosemi & .bdf &  & \Checkmark{} & \Checkmark{} & \Checkmark{} & \tabularnewline
\hline 
Eyelink & txt & .asc &  & \Checkmark{} &  & \Checkmark{} & \tabularnewline
\hline 
European Data Format & edf & .edf & European Data Format & \Checkmark{} & \Checkmark{} & \Checkmark{} & \tabularnewline
\hline 
Philips Scanphyslog & txt & .log & Philips & \Checkmark{} &  &  & \tabularnewline
\hline 
SMI & txt & .txt & SensoMotoric Instruments & \Checkmark{} & \Checkmark{} & \Checkmark{} & \tabularnewline
\hline 
ViewPoint & txt & .txt & Arrington Research & \Checkmark{} & \Checkmark{} & \Checkmark{} & \tabularnewline
\hline 
\end{tabular}

Note: Automarkers means no channel number has to be specified because
markers are always at the same place.

\section{GUI}

\textit{Contributed by Gabriel Gräni.}

\subsection{Matlabbatch: Getting started}
\begin{enumerate}
\item Add the trunk folder to the matlab path
\item Type pspm\_init into the command window (after the execution of the
command the folders pspm\_cfg and matlabbatch should be added to the
matlab path)
\item Start matlabbatch by the typing cfg\_ui into the command window
\item If the item PsPM exists in the menu bar of matlabbatch you can skip
steps 5 to 7 and continue at step 8
\item Select --> File --> Add Application
\item Navigate to the folder pspm\_cfg on the left hand side of the window
and select the file pspm\_cfg.m on the right hand side --> Press
the button Done
\item A new item, called PsPM, will appear in the upper menu bar.
\item By selecting PsPM the desired action can be selected (at the moment,
there is only Data Preparation --> \{Import, Trim\} available) 
\end{enumerate}

\subsubsection{Example Function: Trim}

This example demonstrates how matlabbatch can be used to execute a
function. For all other functions matlabbatch behaves in the same
manner.
\begin{itemize}
\item Select a file by pressing the Select Files Button (under Datafile)
\item Select Reference and choose an item in the lower part of the window
\item Fill in the desired values in the fields which are marked with \textquotedbl <-X\textquotedbl{}
\item After you have chosen a file and filled in all values correctly, you
will see a green arrow on the upper left part of the window
\item By pressing on the green arrow the selected file will be trimmed according
to the filled in values 
\end{itemize}

\subsection{Matlabbatch: How to}

\subsubsection{Preliminaries}
\begin{itemize}
\item Add folder of matlabbatch to the matlab path
\item Add first application and then load the batch in order to execute
a function 
\end{itemize}

\subsubsection{Some notes for creating a new application}
\begin{itemize}
\item Leafs (items) are specified first
\item Assigning child items to .val or .values fields of their parent items
\item Root node of a tree is specified last
\item Some examples of items: 

\begin{itemize}
\item cfg\_item: \\
\texttt{}
\begin{lstlisting}
item1= cfg_item; % Defines generic configuration item 
item1.name= 'Def 1'; % The display name 
item1.tag = 'def1'; % The name appearing in the harvested job 
% structure. This name must be unique 
% among all items in the val field of the 
% superior node 
item1.val = {true}; % Value of item (optional) 
item1.help = {'Help...'}; % Help text 
\end{lstlisting}
\item cfg\_entry: \\
\texttt{}
\begin{lstlisting}
entry1 = cfg_entry; % Defines entry configuration item 
entry1.name = 'Input'; 
entry1.tag = 'input'; 
entry1.strtye = 'r'; % Type of values which can be entered 
entry1.num = [1 1]; % Expected dimension of the input 
entry1.help = {'Help...'}; 
\end{lstlisting}
\item cfg\_choice: \\
\texttt{}
\begin{lstlisting}
choice = cfg_item; % Defines choice configuration item 
choice.name = 'Choice';
choice.tag = 'choice';
choice.values = {item1, entry1}; % Defines which items will be 
% selectable in the choice menu. 
choice.help = {'Help...'}; 
\end{lstlisting}
\item cfg\_exbranch: \\
\texttt{}
\begin{lstlisting}
fct = cfg_exbranch; % Defines the branch that has information 
% about how to run this module fct.name = 'Trim'; 
fct.tag = 'trim'; 
fct.val = {choice}; % The items that belong to this branch. 
% All items must be filled before this 
% branch can run or produce virtual 
% outputs 
\end{lstlisting}
\texttt{}~\\
\texttt{}
\begin{lstlisting}
fct.prog = @cfg_run_fct; % A function handle that will be called 
% with the harvested job to run the 
% computation 
trim.vout = @cfg_vout_fct; % A function handle that will be 
% called with the harvested job to 
% determine virtual outputs 
trim.help = {Help...'}; 
\end{lstlisting}
\item There exists a number of other item classes. Here is a list of the
most important classes: cfg\_item, cfg\_entry, cfg\_choice, cfg\_menu,
cfg\_exbranch, cfg\_files, cfg\_branch, cfg\_repeat\\
\\
For more information call the help function in matlab (e.g. help help
cfg\_item)
\end{itemize}
\item Note:\\
The inputs to each module have to be described in a tree-like structure.\\
During data entry, there is no way to change the tree structure based
on input data. Add application to the configuration tree by default
\end{itemize}

\subsubsection{Add application to the configuration tree by default}

In the following it is shown how an application can be added to the
menu bar of matlabbatch by default (without adding it every time matlabbatch
is started)
\begin{itemize}
\item Start matlabbatch and add the appliaction cfg\_confgui in the folder
matlabbatch/cfg\_confgui 
\item Put Generate code into the Module list by selecting ConfGUI --> Generate
code in the menu bar
\item Fill out all the input fields on the right side: 

\begin{itemize}
\item Output filename: This file will contain the whole menu structure,
validity constraints and links to run time code of the appliaction.
\item Output directory: All files which are created by the ConfGUI will
be stored into this directory (chose a directory which is added to
the matlab path)
\item Root node of config: Name of the root node of the appliaction's configuration
tree
\item Options: 

\begin{enumerate}
\item Create Defaults File: Yes
\item Create mlbatch\_appcfg File: Yes
\item Create Code for addpath(): No 
\end{enumerate}
\end{itemize}
\item Finally press the green arrow on the upper left side of the batch
editor
\item As no error occurred 3 new files (\{Output filename\}.m, \{Output
filename\}\_def.m, cfg\_mlbatch\_appcfg.m) should be created and added
into the folder \{Output directory\}.
\item Each time matlabbatch is started, it will search for any cfg\_mlbatch\_appcfg.m
file (this file contains the names of the configuration files) and
will add the corresponding application to the batch editor. 
\end{itemize}

\subsubsection{Add modules to module list}

Example: Module Import and Trim will be added to the module list\texttt{}~\\
\texttt{}~\\
\texttt{}
\begin{lstlisting}
arg1 = 'scr.prep.import_data'; 
arg2 = 'scr.prep.trim'; 
mod_cfg_id1 = cfg_util('tag2mod_cfg_id',arg1); 
mod_cfg_id2 = cfg_util('tag2mod_cfg_id',arg2); 
cjob = cfg_util('initjob'); 
mod_job_id1 = cfg_util('addtojob', cjob, mod_cfg_id1); 
mod_job_id2 = cfg_util('addtojob', cjob, mod_cfg_id2); 
cfg_util('harvest', cjob, mod_job_id1); 
cfg_util('harvest', cjob, mod_job_id2); 
cfg_ui('local_showjob', cfg_ui, cjob); 
\end{lstlisting}


\subsubsection{Changes}
\begin{itemize}
\item In the function private/cfg\_onscreen at line 36 figure(fg); is commented
out in order to prevent the appearance of the GUI for a short time
if the function cfg\_ui('Visible', 'off') is called.
\end{itemize}

\subsection{Matlabbatch: changing help texts and fieldnames }

\subsubsection{File structure of matlabbatch GUI}

There exist two files per function: 1 configuration file and 1 run
file. The configuration file defines the structure of the corresponding
function in the matlabbatch GUI whereas the run file firstly gathers
all entered values and secondly calls the corresponding SCR function.
Both types of files are located in the subfolder pspm\_cfg. The name
of a configuration or a run file consists of two parts. The prefix
of a configuration filename is called pspm\_cfg\_ whereas the filename
of a run file begins with pspm\_cfg\_run. The second part of the filename
is named after the function name (eg. for the function pspm\_import.m
-> pspm\_cfg\_import.m, pspm\_cfg\_run\_import.m).

\subsubsection{Edit help texts and fieldname}

In order to change any help text or fieldname in a matlabbatch GUI
function the corresponding configuration file has to be opened. For
each item in a matlabbatch GUI function a struct variable which contains
several struct fields is defined in the configuration file. 
\begin{itemize}
\item Help text The field .help defines the help text of the item which
can be edited in order to change the help text. As soon as matlabbatch
has been closed and opened again, the changes in the help text will
be visible in matlabbatch GUI.
\item Fieldname The fieldname of an matlabbatch GUI item is defined by the
struct field .tag . In case a fieldname of an item should be changed
be careful to verify if no other item, which has the same root node,
hold the same fieldname. Otherwise matlabbatch will not work properly.
After the fieldname of an item has been changed the run file (pspm\_cfg\_run\_functionname.m)
of the corresponding function has to be adapted as well in order to
ensure that the function call in the run file is done properly.
\end{itemize}

\subsection{Recommendations for GUI development}

\subsubsection{Mlapp}

Mlapp is the new format for developing GUI in Matlab. The old GUIDE
based GUI will lose support from Mathworks in a future release. Therefore,
it is recommended to move from the old GUIDE to the new Mlapp.

The .mlapp file can be created natively through the new Matlab GUI
guide. Alternatively, it can be generated by converting the classic
GUIDE based .fig file through the feature \textit{migration}.

\subsubsection{Style}

The new GUI is currently using the colour \#7f2534 for stylishing.
The main typeface for UI design is Verdana, which is supported in
Windows, macOS and Linux.

\section{Test Environment}

\textit{Contributed by Linus Rüttimann \& Tobias Moser.}

\subsection{Unittest: General implementation}

In PsPM the Matlab Unit Testing Framework is used for testing of functions.
For each tested function there is a Matlab class with the name \textquoteleft functionname\_test\textquoteright ,
which contains the unittests for that specific function. Additionally
there is a documentation page for each of the test classes, where
information about the unittests can be found.\\
\\
To run the unittests of a test class, an object of the class has to
be created:\\
\\
testCase = functionname\_test\\
\\
where \textquoteleft testCase\textquoteright{} is an arbitrary object
name and \textquoteleft funtionname\_test\textquoteright{} is the
name of a test class. Then all the unittest that are contained in
the test class can be run with: \\
\\
testCase.run\\
\\
A specific unittest can be run with:\\
\\
testCase.run(\textquoteleft unittest\_name\textquoteright )\\
\\
Remember that a new test class object must be generated each time
the test class has been changed.

\subsection{Parameterized test classes}

Parmeterized test classes is a feature provided by the Matlab test
case class. A test class is parameterized when it has 
\begin{itemize}
\item Test parameters defined (within the property section)
\item Test methods implementing the defined test parameters
\end{itemize}
Each function implementing test parameters will be called multiple
times with each possible parameter combination (which is determined
by Matlab). Thus parameterized classes allow to write single tests
for different parameter combinations. If one of the following test
cases is a parameterized test class, it will be mentioned accordingly.

\subsection{Testcases: pspm\_align\_channels\_test}

\subsubsection{Information}

Testclass: pspm\_align\_channels\_test\\
Function: {[}sts, data, duration{]} = pspm\_align\_channels(data,
induration)

\subsubsection{Setup}

This test uses data stored in \texttt{ImportTestData/ecg2hb/tpspm\_s102\_s1.mat}

\subsubsection{Testcases}

\subsubsection*{Invalid input}

Function name: invalid\_input(this)\\
Description: Checks for warnings given invalid inputs.

\subsubsection*{Lower optional duration}

Function name: lower\_optional\_duration(this)\\
Description: Passes an optional duration that is less than the maximum
duration of all channels in the input to pspm\_align\_channels.\\
\\
Tests:
\begin{enumerate}
\item Assert that lower optional duration has no effect on the output.
\item Check if all of the returned channels have the same duration.
\end{enumerate}

\subsubsection*{Same optional duration}

Function name: same\_optional\_duration(this)\\
Description: Passes an optional duration that is equal to the maximum
duration of all channels in the input to pspm\_align\_channels and
does the exact same checks as in lower duration case.

\subsubsection*{Higher optional duration}

Function name: higher\_optional\_duration(this)\\
Description: Passes an optional duration that is higher than the maximum
duration of all channels.\\
\\
Tests:
\begin{enumerate}
\item Assert that durations of all returned channels is the same as the
passed optional duration.
\end{enumerate}

\subsubsection*{Max duration is passed in marker channel}

Function name: max\_duration\_is\_given\_in\_events(this)\\
Description: Passes the maximum duration in marker channel to pspm\_align\_channels.\\
\\
Tests:
\begin{enumerate}
\item Assert that all returned channels are aligned to the maximum duration
passed in marker channel.
\end{enumerate}

\subsubsection*{Various case checks}

Function names:
\begin{enumerate}
\item only\_one\_channel\_longer\_others\_same(this)
\item only\_one\_channel\_shorter\_others\_same(this)
\item increasing\_channel\_lengths(this)
\item two\_same\_others\_shorter(this)
\end{enumerate}
Description: In each of these cases check if the returned channels
have the same duration that is equal to the maximum duration of all
input channels.

\subsection{Testcases: pspm\_butter}

\subsubsection{Information}

Testclass: pspm\_butter\_test \\
Function: {[}sts, b, a{]} = pspm\_butter(order, freqratio, pass)

\subsubsection{Testcases}

\subsubsection*{Invalid input}

Function name: invalid\_input(this) \\
Description: Checks for warnings, if the input arguments are invalid
and if the Signal processing toolbox is installed. \\
\\
Tests: 

\begin{tabular}{|c|c|}
\hline 
Input & Expected warning\tabularnewline
\hline 
\hline 
pspm\_butter() {[}no input{]} & ID:invalid\_input\tabularnewline
\hline 
pspm\_butter(1,1,'abc') {[}pass not equal to 'high' or 'low'{]} & ID:invalid\_input\tabularnewline
\hline 
pspm\_butter(2,1) {[}'Signal processing toolbox is missing' \#1{]} & ID:toolbox\_missing\tabularnewline
\hline 
pspm\_butter(1,1) {[}'Signal processing toolbox is missing' \#2{]} & ID:toolbox\_missing\tabularnewline
\hline 
\end{tabular}

\subsection{Testcases: pspm\_bf\_test}

\subsubsection{Information}

Testclass: pspm\_bf\_test\\
Function: {[}bs, x{]} = pspm\_bf\_<specific function name>

\subsubsection{Setup}

This test class is parameterized.

\subsubsection*{Method setup parameters}

These parameters define which function should be tested\textbf{.}\\
\begin{tabular}{|c|>{\raggedright}p{10cm}|}
\hline 
\textbf{Basis function} & Specifies the basis functions to test (without the 'pspm\_bf\_' prefix).
The current basis function to test is then called via this.bf();\tabularnewline
\hline 
\end{tabular}

\subsubsection*{Test parameters}

These are parameters which define what kind of data or option should
be passed to each basis function.\\
\begin{tabular}{|c|>{\raggedright}p{10cm}|}
\hline 
\textbf{Time res log} & Specifies for the basic test different time resolutions (argument
'td') which a basis function should be able to handle (as long as
td <= duration). The values are logarithmic and have to be translated
before passed to the basis function.\tabularnewline
\hline 
\end{tabular}

\subsubsection{Testcases}

\subsubsection*{Invalid input}

Function name: invalid\_input(this) \\
Description: Checks for warnings, if the input arguments are invalid.
\\
\\
Tests: \\
\begin{tabular}{|>{\centering}p{6cm}|c|}
\hline 
Input & Expected warning\tabularnewline
\hline 
\hline 
this.bf() {[}no parameters{]} & ID:invalid\_input\tabularnewline
\hline 
this.bf(dur+1) {[}pass 'td' > duration of function{]} & ID:invalid\_input\tabularnewline
\hline 
this.bf(0) {[}invalid time resolution{]} & ID:invalid\_input\tabularnewline
\hline 
\end{tabular}

\subsubsection*{Basic}

Function name: test\_basic(this, time\_res\_log)\\
Description: Test for different requirements to verify whether the
current basis function is valid or not.\\
\\
Tests:
\begin{enumerate}
\item Test with td = 0.1, verify that no warning is issued and determine
the duration
\item Test with td = 0.01 and check if the new duration is equal to the
duration calculated before.
\item Test if function runs through without warning and that the time vector
begins at <= 0.
\item Test if the function runs through without warning with td = 10\textasciicircum time\_res\_log
(as long as td < duration)
\end{enumerate}

\subsection{Testcases: pspm\_convert\_unit}

\subsubsection{Information}

Testclass: pspm\_convert\_unit\_test\\
Function: {[}sts, converted{]} = pspm\_convert\_unit(data, from, to)

\subsubsection{Setup}

\subsubsection*{Constants}
\begin{itemize}
\item inch\_to\_cm = 2.54
\end{itemize}

\subsubsection{Testcases}

\subsubsection*{Invalid input}

Function name: invalid\_input(testCase)\\
Description: Pass invalid from or to metrics and check if warnings
are issued.

\subsubsection*{Valid input}

Function name: valid\_input(this)\\
Description: Pass various valid inputs and compare results to manually
calculated ones.\\
\\
Tests:
\begin{enumerate}
\item If empty input data is passed, result is also empty.
\item Convert single cm value to m.
\item Various unit conversion checks:
\begin{enumerate}
\item Conversion between same units (cm to cm)
\item mm to km and km to mm conversions
\item inch to cm conversions
\end{enumerate}
\item Negative value conversions
\item Convert single dimensional array with multiple elements.
\item Convert each element in 3D array.
\end{enumerate}

\subsection{Testcases: pspm\_ecg2hb}

\subsubsection{Information}

Testclass: pspm\_ecg2hb\_test \\
\\
Function: {[}sts,pt\_debug{]} = pspm\_ecg2hb(fn, chan, options)

\subsubsection{Setup}

\subsubsection*{Constants}
\begin{itemize}
\item testdata\{0\}.chan\_struct = struct('nr', 1, 'name', 'ecg');
\item testdata\{0\}.filename = 'ImportTestData\textbackslash ecg2hb\textbackslash test\_ecg77.mat';
\item testdata\{0\}.num\_channels = 1
\item testdata\{1\}.chan\_struct = struct('nr', 3, 'name', 'ecg');
\item testdata\{1\}.filename = 'ImportTestData\textbackslash ecg2hb\textbackslash tpspm\_s102\_s1.mat';
\item testdata\{1\}.num\_channels = 5
\item backup\_suffix = '\_backup';
\item options = struct('semi', 0);
\end{itemize}

\subsubsection{Testcases}

\subsubsection*{Invalid input arguments}

Function name: invalid\_input(this)\\
\\
Description: Pass invalid input arguments and check if the warnings
are as expected.\\
\\
Tests: \\
\\
\begin{tabular}{|>{\centering}p{10cm}||>{\centering}p{5cm}|}
\hline 
Input & Expected warning\tabularnewline
\hline 
\hline 
pspm\_ecg2hb() {[}no arguments{]} & ID:invalid\_input \tabularnewline
\hline 
\hline 
pspm\_ecg2hb(1) {[}invalid file name{]} & ID:invalid\_input\tabularnewline
\hline 
\hline 
pspm\_ecg2hb(this.fn, 'bla') {[}invalid channel (text){]} & ID:invalid\_input \tabularnewline
\hline 
\hline 
pspm\_ecg2hb(this.fn, 1) {[}invalid channel type{]} & ID:not\_allowed\_channeltype\tabularnewline
\hline 
\hline 
o.twthresh = 'bla'; pspm\_ecg2hb(this.fn, this.chan.nr, o) {[}invalid
twthresh (text){]} & ID:invalid\_input \tabularnewline
\hline 
\hline 
o.minHR = 202; pspm\_ecg2hb(this.fn, this.chan.nr, o) {[}invalid minHR
(> default\_maxHR){]} & ID:invalid\_input \tabularnewline
\hline 
\hline 
o.minHR = 202; o.maxHR = 19; pspm\_ecg2hb(this.fn, this.chan.nr, o)
{[}invalid minHR > maxHR{]} & ID:invalid\_input \tabularnewline
\hline 
\hline 
o.maxHR = 19; pspm\_ecg2hb(this.fn, this.chan.nr, o) {[}invalid maxHR
(< default\_minHR){]} & ID:invalid\_input \tabularnewline
\hline 
\hline 
o.debugmode = 5; pspm\_ecg2hb(this.fn, this.chan.nr, o) {[}invalid
debugmode (not in {[}0,1{]}){]} & ID:invalid\_input \tabularnewline
\hline 
\hline 
o.semi = 5; pspm\_ecg2hb(this.fn, this.chan.nr, o) {[}invalid semi
(not in {[}0,1{]}){]} & ID:invalid\_input \tabularnewline
\hline 
\end{tabular}

\subsubsection*{Valid input arguments}

Function name: valid\_input(this)\\
\\
Description: Pass valid input arguments and check if there are no
warnings.\\
\\
Tests: \\
\\
\begin{tabular}{|c|c|}
\hline 
Input & Expected warning\tabularnewline
\hline 
\hline 
pspm\_ecg2hb(this.fn, this.chan.nr, this.options) & -\tabularnewline
\hline 
pspm\_ecg2hb(this.fn, this.chan.name, this.options) & -\tabularnewline
\hline 
this.test\_added\_data() & -\tabularnewline
\hline 
\end{tabular}

\subsubsection{Other Methods}

\subsubsection*{Test for added data}

Function name: test\_added\_data()\\
\\
Description: Check if added hb channels show an expected behaviour.\\
\\
Tests (for each channel):\\
\\
\begin{tabular}{|c|c|}
\hline 
Tested Value & Expected Value\tabularnewline
\hline 
\hline 
Sampling rate & 1\tabularnewline
\hline 
Unit & 'events'\tabularnewline
\hline 
Channel type & 'hb'\tabularnewline
\hline 
Amount of data points in data & > 1\tabularnewline
\hline 
Heartbeat indices are monotonically increasing & True\tabularnewline
\hline 
Maximum number of heartbeats per second & < 5\tabularnewline
\hline 
Data is distributed equally (standard deviation) & < 2s\tabularnewline
\hline 
Time between end of recording and last data point & < 60s\tabularnewline
\hline 
\end{tabular}

\subsection{Testcases: pspm\_filtfilt}

\subsubsection{Information}

Testclass: pspm\_filtfilt\_test \\
Function: y = pspm\_filtfilt(b,a,x)

\subsubsection{Testcases}

\subsubsection*{Invalid input}

Function name: invalid\_input(this) \\
Description: Checks for warnings, if the input arguments are invalid.
\\
\\
Tests: 

\begin{tabular}{|c|c|}
\hline 
Input & Expected warning\tabularnewline
\hline 
\hline 
pspm\_filtfilt() {[}no input{]} & ID:invalid\_input\tabularnewline
\hline 
pspm\_filtfilt({[}1:10{]},{[}1:20{]},{[}1:10{]}) {[}data length less
than 3 times filter order{]} & ID:invalid\_input\tabularnewline
\hline 
\end{tabular}

\subsection{Testcases: pspm\_find\_channel}

\subsubsection{Information}

Testclass: pspm\_find\_channel\_test \\
\\
Function: chan = pspm\_find\_channel(headercell, chantype) 

\subsubsection{Testcases}

\subsubsection*{Invalid input arguments}

Function name: invalid\_inputargs(this) \\
\\
Description: Checks for warnings, if the input arguments are invalid.
\\
\\
Setup: \\
\\
\texttt{headercell = \{'heart', 'scr', 'pupil'\};}\\
\\
Tests: \\
\\
\begin{tabular}{|c|c|}
\hline 
Input & Expected warning\tabularnewline
\hline 
\hline 
pspm\_find\_channel('str','scr') {[}no headercell{]} & ID:invalid\_input \tabularnewline
\hline 
pspm\_find\_channel(headercell, 'str') & ID:not\_allowed\_channeltype\tabularnewline
\hline 
pspm\_find\_channel(headercell, 4) {[}no string chantype{]} & ID:invalid\_input\tabularnewline
\hline 
\end{tabular}

\subsubsection*{Valid Input Arguments}

Function name: valid\_inputargs(this) \\
\\
Description: Checks for correct return value if the input arguments
are valid \\
\\
Setup: \\
\\
\texttt{headercell = \{'heart', 'scr', 'pupil', 'mark', 'gsr', 'eda'\};}\\
\\
Tests: \\
\\
\begin{tabular}{|>{\centering}p{6cm}|c|>{\centering}p{6cm}|}
\hline 
Input & Exp. Output & Expected warning\tabularnewline
\hline 
\hline 
pspm\_find\_channel(headercell, 'pupil') & 3 & \tabularnewline
\hline 
pspm\_find\_channel(headercell, 'resp') & 0 & ID:no\_matching\_channels\tabularnewline
\hline 
pspm\_find\_channel(headercell, 'scr') & -1 & ID:multiple\_matching\_channels\tabularnewline
\hline 
pspm\_find\_channel(headercell, \{'mark', 'str', 'bla'\}) & 4 & \tabularnewline
\hline 
pspm\_find\_channel(headercell, \{'call', 'str', 'me'\}) & 0 & no matching channel, but no warning\tabularnewline
\hline 
pspm\_find\_channel(headercell, \{'scr', 'gsr', 'eda'\}) & -1 & multiple matching channels, but no warning\tabularnewline
\hline 
\end{tabular}

\subsection{Testcases: pspm\_extract\_segments}

\subsubsection{Information}

Testclass: pspm\_extract\_segments\_test\\
Function: {[}sts, out{]} = pspm\_extract\_segments(varargin)

\subsubsection{Setup}

This test class is parameterized. For manual mode tests, the test
data is generated by the function itself and when needed, files will
be written to testdatafile<variable\_nr>.mat. For auto mode tests,
the test data must be in \texttt{ImportTestData/fitted\_models} folder
with names as specified in the tests.

\subsubsection*{Test parameters}

These are parameters which define what kind of data should be passed
to \texttt{pspm\_extract\_segments} in auto mode tests and which options
should be set.
\begin{itemize}
\item nan\_output: This option defines whether the user wants to output
the NaN ratios of the trials for each condition. If so, we values
can be printed on the screen (on MATLAB command window) or written
to a created file.
\item nan\_ratio: Defines ratio of NaN values in the generated test data
\item nr\_trail: Number of trails in the generated test data
\end{itemize}

\subsubsection{Testcases}

\subsubsection*{Invalid input}

Function name: invalid\_input(this) \\
Description: Checks for warnings, if the input arguments are invalid.

Tests: \\
\begin{tabular}{|>{\centering}p{6cm}|c|}
\hline 
Input & Expected warning\tabularnewline
\hline 
\hline 
pspm\_extract\_segments() & ID:invalid\_input\tabularnewline
\hline 
pspm\_extract\_segments('a','b') & ID:invalid\_input\tabularnewline
\hline 
pspm\_extract\_segments('manual',fn,0) & ID:invalid\_input\tabularnewline
\hline 
pspm\_extract\_segments('manual',struct('a',10),0,timing) & ID:invalid\_input\tabularnewline
\hline 
pspm\_extract\_segments('manual',{[}1,3{]},logical(32),timing) & ID:invalid\_input\tabularnewline
\hline 
pspm\_extract\_segments('manual',fn,'a',timing) & ID:invalid\_input\tabularnewline
\hline 
pspm\_extract\_segments('manual',fn,\{'a'\},timing) & ID:invalid\_input\tabularnewline
\hline 
pspm\_extract\_segments('auto',\{1\}) & ID:invalid\_input\tabularnewline
\hline 
pspm\_extract\_segments('auto','some') & ID:invalid\_input\tabularnewline
\hline 
\end{tabular}

\subsubsection*{Test manual mode with indicated length}

Function name: test\_manual\_length(this,nr\_trial,nan\_ratio)\\
Desctiption: Checks for equality of produced segments by pspm\_extract\_segments
with manually computed segments\\
\\
Tests:
\begin{enumerate}
\item Generate segments form test data.
\item Test if function call wirked WarningFree
\item Test if variable 'segments' existis in output
\item Test if correct number of segments were produced
\item Test each segment holds correct data 
\end{enumerate}

\subsubsection*{Test manual mode with durations}

Function name: test\_manual\_duration(this,nr\_trial,nan\_ratio)\\
Desctiption: Checks for equality of produced segments by pspm\_extract\_segments
with manually computed segments\\
\\
Tests:
\begin{enumerate}
\item Generate segments form test data.
\item Test if function call wirked WarningFree
\item Test if variable 'segments' existis in output
\item Test if correct number of segments were produced
\item Test each segment holds correct data
\end{enumerate}

\subsubsection*{Test auto mode with GLM using marker onsets}

Function name: \texttt{test\_auto\_mode\_glm\_with\_markers(this)}\\
Description: Runs \texttt{pspm\_extract\_segments} with a particular
GLM model stored in \texttt{ImportTestData/fitted\_models} and compares
the results to manually calculated results.\\
\\
Tests:
\begin{enumerate}
\item Test if length of the returned cell array (from now on called \texttt{segments})
is the same as the number of conditions
\item Test if shape of data arrays in each element of \texttt{segments}
agrees with the passed durations and number of onsets.
\item Test if \texttt{segments.trial\_idx} agrees with input data.
\item Test if statistics calculated manually from \texttt{segments.data}
is the same as \texttt{segments.mean} and \texttt{segments.std}.
\item Compute each statistic field in each element of \texttt{segments}
manually using the input data and compare the results to \texttt{segments}.
\end{enumerate}

\subsubsection*{Test auto mode with GLM using second onsets}

Function name: \texttt{test\_auto\_mode\_glm\_with\_seconds(this)}\\
Description: Do the exact same tests as in \texttt{test\_auto\_mode\_glm\_with\_markers}
but this time using seconds to specify onsets.

\subsubsection*{Test auto mode with DCM}

Note: Since in DCM case onsets are calculated using trial start and
end seconds of DCM trials, there is no second/marker distinction in
DCM test.\\
Function name: \texttt{test\_auto\_mode\_dcm(this)}\\
Description: Runs \texttt{pspm\_extract\_segments} with a particular
DCM model stored in \texttt{ImportTestData/fitted\_models} and compares
the results to manually calculated results. In order to get meaningful
condition statistic information this test function assigns the same
trial name to certain groups of trials.\\
\\
Tests: Do the exact same tests as in \texttt{test\_auto\_mode\_glm\_with\_markers}
by adapting the computation steps to DCM case.

\subsection{Testcases: pspm\_find\_sounds}

\subsubsection{Information}

Testclass: pspm\_find\_sounds\_test\\
Function: {[}sts, infos{]} = pspm\_find\_sounds(file, options)

\subsubsection{Setup}

This test class is parameterized. The test data is generated by the
function itself and when needed, files will be written to testdatafile<variable\_nr>.mat.

\subsubsection*{Test parameters}

These are parameters which define what kind of data should be passed
to pspm\_find\_sounds and which options should be set.\\
\begin{tabular}{|c|>{\raggedright}p{10cm}|}
\hline 
\textbf{Channel output} & Specifies whether 'all' found markers or only 'corrected' markers
should be returned.\tabularnewline
\hline 
\textbf{Max delay} & Varies the max delay option and defines how far away a marker at most
can be.\tabularnewline
\hline 
\textbf{Min delay} & Varies the min delay option and defines how far away a marker at least
should be.\tabularnewline
\hline 
\textbf{Threshold} & Defines the minimum size of a marker to be recognized as a marker
event. Passed in percent of the maximum amplitude of the recorded
data.\tabularnewline
\hline 
\textbf{Resample} & Defines whether the function should resample (and interpolate) the
data to a higher sample rate in order to get more exact marker findings.\tabularnewline
\hline 
\textbf{Channel action} & Defines whether a newly created marker channel should replace the
existing marker channel or should be added as a new marker channel.\tabularnewline
\hline 
\end{tabular}

\subsubsection{Testcases}

\subsubsection*{Invalid input}

Function name: invalid\_input(this) \\
Description: Checks for warnings, if the input arguments are invalid.
\\
\\
Tests: \\
\begin{tabular}{|>{\centering}p{6cm}|c|}
\hline 
Input & Expected warning\tabularnewline
\hline 
\hline 
pspm\_find\_sounds('') & ID:file\_not\_found\tabularnewline
\hline 
pspm\_find\_sounds(fn) {[}invalid pspm file{]} & ID:invalid\_input\tabularnewline
\hline 
pspm\_find\_sounds(fn) {[}pspm file without a 'snd' channel{]} & ID:no\_sound\_chan\tabularnewline
\hline 
pspm\_find\_sounds(fn, o) {[}invalid values for positive integer fields{]} & ID:invalid\_input\tabularnewline
\hline 
pspm\_find\_sounds(fn, o) {[}invalid values for positive numeric fields{]} & ID:invalid\_input\tabularnewline
\hline 
pspm\_find\_sounds(fn, o) {[}invalid values for logic fields{]} & ID:invalid\_input\tabularnewline
\hline 
pspm\_find\_sounds(fn, o) {[}invalid channel ids for channel fields{]} & ID:out\_of\_range\tabularnewline
\hline 
pspm\_find\_sounds(fn, o) {[}enabled diagnostics without a marker
channel{]} & ID:no\_marker\_chan\tabularnewline
\hline 
pspm\_find\_sounds(fn, o) {[}invalid values for channelaction{]} & ID:invalid\_input\tabularnewline
\hline 
pspm\_find\_sounds(fn, o) {[}invalid values for roi{]} & ID:invalid\_input\tabularnewline
\hline 
pspm\_find\_sounds(fn, o) {[}maxdelay < mindelay{]} & ID:invalid\_input\tabularnewline
\hline 
\end{tabular}

\subsubsection*{Test add channel}

Function name: test\_add\_channel(this, channeloutput, max\_delay,
resample, channelaction)\\
Description: Test add channel with different options. Diagnostics
is always enabled, Channel output, Max delay, Resample and Channel
action are varied. Once pspm\_find\_sounds is complete, the function
tests if the returned data has the expected format.\\
\\
Tests:
\begin{enumerate}
\item Generate data with channel 'snd' and 'marker'; and count amount of
reference markers
\item Set

\begin{enumerate}
\item options according to test parameters
\item diagnostics to 1
\end{enumerate}
\item Test if function runs through without warning
\item Test if returned data has the correct format
\item Test if channels has been added or replaced
\item Test if added channel has correct amount of data
\end{enumerate}

\subsubsection*{Test region count}

Function name: test\_region\_count(this)\\
Description: Test region of interest in combination with expected
sound count.\\
\\
Tests:
\begin{enumerate}
\item Generate data with channel 'snd' and 'marker'
\item Test if function finds the function finds all markers in the whole
file
\item Test if function finds all the markers in the whole file with initial
threshold 1
\item Test if function finds half of the markers in half of the file
\end{enumerate}

\subsubsection*{Test threshold}

Function name: test\_threshold(this, threshold)\\
Description: Vary the threshold option and test whether the functions
returnes the expected data.\\
\\
Tests:
\begin{enumerate}
\item Generate data with channel 'snd' and 'marker'
\item Set

\begin{enumerate}
\item threshold according to test parameter
\item diagnostics to 1
\end{enumerate}
\item Test if function runs through without warning
\item Test if returned data has the correct format
\end{enumerate}

\subsubsection*{Test plot}

Function name: test\_plot(this, threshold)\\
Description: Test if the plot functions returne the expected data
and runs through without warning.\\
\\
Tests:
\begin{enumerate}
\item Generate data with channel 'snd' and 'marker'
\item Set

\begin{enumerate}
\item plot to 1
\item diagnostics to 1
\end{enumerate}
\item Test if function runs through without warning
\item Test if returned data has the correct format
\end{enumerate}

\subsection{Testcases: pspm\_find\_valid\_fixations}

\subsubsection{Information}

Testclass: pspm\_find\_valid\_fixations\_test\\
Function: {[}sts, out\_file{]} = pspm\_find\_valid\_fixations(fn,
options)

\subsubsection{Setup}

This test class is parameterized. The test data is generated by the
function itself and when needed, files will be written to testdatafile<variable\_nr>.mat.

\paragraph*{Test parameters}

These are parameters which define what kind of data should be passed
to pspm\_find\_valid\_fixations and which options should be set.\\
\begin{tabular}{|c|>{\raggedright}p{10cm}|}
\hline 
\textbf{Distance} & Used for gaze validation; defines the distance between eyes and screen.\tabularnewline
\hline 
\textbf{Aspect used} & Used for gaze validation; defines the aspect ratio set in the software.\tabularnewline
\hline 
\textbf{Aspect actual} & Used for gaze validation; defines the aspect ratio of the hardware.\tabularnewline
\hline 
\textbf{Screen size} & Used for gaze validation; defines the size of the screen in inches.\tabularnewline
\hline 
\textbf{Eyes} & Is used for data generation and tells the function for which eyes
data should be generated.\tabularnewline
\hline 
\textbf{Channel action} & Defines whether to 'add' or 'replace' existing channels.\tabularnewline
\hline 
\textbf{Newfile} & Defines whether to create a new file or extend the existing file.\tabularnewline
\hline 
\textbf{Overwrite} & Defines whether to overwrite the existing file or not.\tabularnewline
\hline 
\textbf{Interpolate} & Defines whether to interpolate NaN values in validated channels or
not.\tabularnewline
\hline 
\textbf{Missing} & Defines whether to create a channel which holds information about
which positions have been set to NaN (and may have been interpolated
afterwards).\tabularnewline
\hline 
\textbf{Work eye} & Defines which eyes should be used for fixation validation.\tabularnewline
\hline 
\textbf{Work chans} & Defines which channels should be set to NaN during invalid fixations.\tabularnewline
\hline 
\end{tabular}

\subsubsection{Testcases}

\subsubsection*{Invalid input}

Function name: invalid\_input(this) \\
Description: Checks for warnings, if the input arguments are invalid.
\\
\\
Tests: \\
\begin{tabular}{|>{\centering}p{6cm}|c|}
\hline 
Input & Expected warning\tabularnewline
\hline 
\hline 
pspm\_find\_valid\_fixations() & ID:invalid\_input\tabularnewline
\hline 
pspm\_find\_valid\_fixations('a') & ID:invalid\_input\tabularnewline
\hline 
pspm\_find\_valid\_fixations(fn, options) {[}invalid options.validate\_fixations{]} & ID:invalid\_input\tabularnewline
\hline 
pspm\_find\_valid\_fixations(fn, options) {[}invalid options.box\_degree{]} & ID:invalid\_input\tabularnewline
\hline 
pspm\_find\_valid\_fixations(fn, options) {[}invalid options.screen\_settings{]} & ID:invalid\_input\tabularnewline
\hline 
pspm\_find\_valid\_fixations(fn, options) {[}missing fields for options.screen\_settings{]} & ID:invalid\_input\tabularnewline
\hline 
pspm\_find\_valid\_fixations(fn, options) {[}invalid options.aspect\_actual{]} & ID:invalid\_input\tabularnewline
\hline 
pspm\_find\_valid\_fixations(fn, options) {[}invalid options.aspect\_used{]} & ID:invalid\_input\tabularnewline
\hline 
pspm\_find\_valid\_fixations(fn, options) {[}invalid options.bitmap{]} & ID:invalid\_input\tabularnewline
\hline 
pspm\_find\_valid\_fixations(fn, options) {[}invalid options.display\_size{]} & ID:invalid\_input\tabularnewline
\hline 
pspm\_find\_valid\_fixations(fn, options) {[}invalid options.display\_size{]} & ID:invalid\_input\tabularnewline
\hline 
pspm\_find\_valid\_fixations(fn, options) {[}invalid options.fixation\_point{]} & ID:invalid\_input\tabularnewline
\hline 
pspm\_find\_valid\_fixations(fn, options) {[}invalid options.channel\_action{]} & ID:invalid\_input\tabularnewline
\hline 
pspm\_find\_valid\_fixations(fn, options) {[}invalid options.newfile{]} & ID:invalid\_input\tabularnewline
\hline 
pspm\_find\_valid\_fixations(fn, options) {[}invalid options.overwrite{]} & ID:invalid\_input\tabularnewline
\hline 
pspm\_find\_valid\_fixations(fn, options) {[}invalid options.interpolate{]} & ID:invalid\_input\tabularnewline
\hline 
pspm\_find\_valid\_fixations(fn, options) {[}invalid options.missing{]} & ID:invalid\_input\tabularnewline
\hline 
pspm\_find\_valid\_fixations(fn, options) {[}invalid eyes{]} & ID:invalid\_input\tabularnewline
\hline 
pspm\_find\_valid\_fixations(fn, options) {[}invalid options.channels{]} & ID:invalid\_input\tabularnewline
\hline 
\end{tabular}

\subsubsection*{Test work chans}

Function name: test\_work\_chans(this, work\_chans)\\
Description: Tests whether the option 'channels' actually works on
the specified channels or not.\\
\\
Tests:
\begin{enumerate}
\item Generate data with distance 500, aspect\_used 16:9, aspect\_actual
4:3, screen\_size 20 and eyes 'lr'
\item Set options with

\begin{enumerate}
\item overwrite = 1
\item channels = work\_chans
\item channel\_action = 'add'
\end{enumerate}
\item Test if function runs through without warning
\item Test if sts equals 1
\item Test if specified work\_chans are added as new processed channels
\end{enumerate}

\subsubsection*{Test work eye}

Function name: test\_work\_eye(this, work\_eye)\\
Description: Test whether the option 'eyes' actually works on the
specified eyes or not.\\
\\
Tests:
\begin{enumerate}
\item Generate data with distance 500, aspect\_used 16:9, aspect\_actual
4:3, screen\_size 20 and eyes 'lr'
\item Set options with

\begin{enumerate}
\item overwrite = 1
\item eyes = work\_eye
\item channel\_action = 'add'
\end{enumerate}
\item Test if function runs through without warning
\item Test if sts equals 1
\item Test if specified eyes have been processed accordingly and test if
not specified eyes have ignored.
\end{enumerate}

\subsubsection*{Test missing}

Function name: test\_missing(this, missing)\\
Description: Test whether for each a a new missing channel is created
if missing is specified as true.\\
\\
Tests:
\begin{enumerate}
\item Generate data with distance 500, aspect\_used 16:9, aspect\_actual
4:3, screen\_size 20 and eyes 'lr'
\item Set options with

\begin{enumerate}
\item overwrite = 1
\item missing = missing
\item channel\_action = 'add'
\end{enumerate}
\item Test if function runs through without warning
\item Test if sts equals 1
\item Depending on the status of 'missing' test if there are any missing
channels or if there is no missing channel
\end{enumerate}

\subsubsection*{Test interpolate}

Function name: test\_interpolate(this, interpolate)\\
Description: Test whether data is interpolated during periods which
are set to NaN by the function.\\
\\
Tests:
\begin{enumerate}
\item Generate data with distance 500, aspect\_used 16:9, aspect\_actual
4:3, screen\_size 20 and eyes 'lr'
\item Set options with

\begin{enumerate}
\item overwrite = 1
\item interpolate = interpolate
\item channel\_action = 'add'
\end{enumerate}
\item Test if function runs through without warning
\item Test if sts equals 1
\item Depending on the status of 'interpolate' test whether there are some
NaN values or if NaN periods have been interpolated accordingly.
\end{enumerate}

\subsubsection*{Test overwrite}

Function name: test\_overwrite(this, overwrite)\\
Description: Test if files are overwritten, if specified with 'overwrite'
option.\\
\\
Tests:
\begin{enumerate}
\item Generate data with distance 500, aspect\_used 16:9, aspect\_actual
4:3, screen\_size 20 and eyes 'lr'
\item Set options with

\begin{enumerate}
\item overwrite = 1
\item interpolate = interpolate
\item channel\_action = 'add'
\end{enumerate}
\item Test if function runs through without warning
\item Test if sts equals 1
\item Test if file has been overwritten or not (tests, if there are any
new channels).
\end{enumerate}

\subsubsection*{Test channel action}

Function name: test\_channel\_action(this, channel\_action)\\
Description: Test if channels are added or replaced (according to
channel\_action).\\
\\
Tests:
\begin{enumerate}
\item Generate data with distance 500, aspect\_used 16:9, aspect\_actual
4:3, screen\_size 20 and eyes 'lr'
\item Set options with

\begin{enumerate}
\item overwrite = 1
\item channel\_action = channel\_action
\end{enumerate}
\item Test if function runs through without warning
\item Test if sts equals 1
\item Test if channels have been added or replaced (tests, if there are
any new channels).
\end{enumerate}

\subsubsection*{Test newfile}

Function name: test\_newfile(this, newfile)\\
Description: Test whether the output is written to a newfile or to
the input file.\\
\\
Tests:
\begin{enumerate}
\item Generate data with distance 500, aspect\_used 16:9, aspect\_actual
4:3, screen\_size 20 and eyes 'lr'
\item Set options with

\begin{enumerate}
\item overwrite = 1
\item if newfile enabled
\begin{enumerate}
\item search for new file name
\item set options.newfile to new file name
\end{enumerate}
\item if newfile is disabled, set options.newfile to ''
\end{enumerate}
\item Test if function runs through without warning
\item Test if sts equals 1
\item Test if returned outputfile equals the specified newfile or not (depending
on the value of 'newfile')
\end{enumerate}

\subsubsection*{Test gaze validation}

Function name: test\_gaze\_validation(this, distance, screen\_size,
aspect\_actual, aspect\_used, eyes)\\
Description: Test whether gaze validation is done correctly.\\
\\
Tests:
\begin{enumerate}
\item Generate data with the according function parameters
\item Iterate to returned degree values generated by the generation function

\begin{enumerate}
\item set function options
\begin{enumerate}
\item overewrite = 1
\item validate\_fixation =1
\item screen\_settings and distance to function call settings
\item missing = 1
\end{enumerate}
\item depending on the specified degree, test whether function runs through
without warnings or not
\item load outputfile and test if (according to degree expectation) gaze
validation has been done or not
\end{enumerate}
\end{enumerate}

\subsubsection*{Test bitmap validtion}

Function name: test\_bitmap\_validation(this, distance, resolution,
eyes)\\
Description: Test whether bitmap validation is done correctly.\\
\\
Tests:
\begin{enumerate}
\item Generate data with the according function parameters
\item Iterate to returned bitmaps generated by the generation function

\begin{enumerate}
\item set function options
\begin{enumerate}
\item missing = 1
\end{enumerate}
\item ddepending on the specified number of valid fixations in the bitmap,
test whether function runs through without warnings or not
\item load outputfile and test if (according to bitmap expectation) bitmap
validation has been done or not
\end{enumerate}
\end{enumerate}

\subsection{Testcases: pspm\_get\_ecg}

\subsubsection{Information}

Testclass: pspm\_get\_ecg\_test \\
Function: {[}sts, data{]} = pspm\_get\_ecg(import) 

\subsubsection{Testcases}

\subsubsection*{Test}

Function name: test(this) \\
Description: Test if all fields are returned correctly \\
\\
Tests: 
\begin{enumerate}
\item Test if \textquoteleft sts\textquoteright{} is equal 1.
\item Test if data.data is equal import.data
\item Test if data.header.chantype is \textquoteleft ecg\textquoteright{}
\item Test if data.header.units is equal import.units
\item Test if data.header.sr is equal import.sr 
\end{enumerate}

\subsection{Testcases: pspm\_get\_events}

\subsubsection{Information}

Testclass: pspm\_get\_events\_test \\
Function: {[}sts, import{]} = pspm\_get\_events(import) 

\subsubsection{Testcases}

\subsubsection*{Check warnings}

Function name: check\_warnings(this) \\
Description: Checks for warnings, if the field \textquoteleft .markers\textquoteright{}
is missing or contains invalid content. \\
\\
Tests: 

\begin{tabular}{|c|c|}
\hline 
Input & Expected warning\tabularnewline
\hline 
\hline 
Missing marker field & ID:nonexistent\_field\tabularnewline
\hline 
import.marker = \textquoteleft foo\textquoteright{} & ID:invalid\_field\_content\tabularnewline
\hline 
\end{tabular}

\subsubsection*{Timestamps}

Function name: timestamps(this) \\
Description: Checks for correct output if the input is timestamp data
\\
\\
Tests:
\begin{enumerate}
\item Test if \textquoteleft sts\textquoteright{} is equal 1.
\item Test if the length of the output data is equal to the length of the
input data 
\end{enumerate}

\subsubsection*{Continuous}

Function name: continuous(this) \\
Description: Checks for correct output if the input is continuous
data \\
\\
Tests:
\begin{enumerate}
\item Perform three tests with different settings

Tests:
\begin{enumerate}
\item Test if \textquoteleft sts\textquoteright{} is equal 1.
\item Test if the length of the field 'markerinfo' is equal to the length
of the output data.
\item Test if the length of the output data is equal to the expected number
of pulses in the input data.
\end{enumerate}
Settings:
\begin{enumerate}
\item flank = \textquoteleft both\textquoteright{} (default)
\item flank = 'both' \& data offset 50
\item flank = \textquoteleft ascending\textquoteright{}
\item flank = \textquoteleft descending\textquoteright{}
\item inverted input signal
\item signal with angular flanks 
\item check with
\end{enumerate}
\item Additional test for setting (b): Test if data offset has been removed
in the output data.
\item Additional test for setting (c) and (d): Test if positions returned
by output data correspond to flank changes in the input data.
\item Test if markerinfo is not set if it has been set before.
\end{enumerate}

\subsection{Testcases: pspm\_get\_eyelink}

\subsubsection{Information}

Testclass: pspm\_get\_eyelink\_test \\
Function: {[}sts, data{]} = pspm\_get\_eyelink(import) 

\subsubsection{Methods}

\subsubsection*{set\_import\_values}

Function : {[}import\_struct, channel typles{]} = set\_import\_values(this)\\
Description: Helperfunction, which creates an import data set and
the expected channel data set 

\subsubsection*{verify\_basic\_data\_structure}

Function name: verify\_basic\_data\_structure(this, data, sourceinfo,
channel\_types)\\
Description: Tests if the returned data structure is valid and match
a given expected pattern.\\
\\
Tests:
\begin{enumerate}
\item Test if all channels are numeric
\item Test if recorded time and date have a valid format
\item Test if blink channels have correct unit
\item Test if pupil channels have either 'diameter' or 'area' as unit
\item Test if channels labeled with 'position' have unit 'pixel'
\item Test if channels labeled with 'blink' have unit 'blink'
\end{enumerate}

\subsubsection{Testcases}

\subsubsection*{test\_multi\_session}

Function name: test\_multi\_session(this)\\
Description: Test if the returned data structure fits into the pattern
of a multi session data set.\\
\\
Tests:
\begin{enumerate}
\item Calls 'set\_import\_values(this)' to get import data set and expected
channel data set 
\item passses returned sets to 'verify\_basic\_data\_structure()'
\end{enumerate}

\subsubsection*{test\_two\_eyes}

Function name: test\_two\_eyes(this)\\
Description: Test if the returned data structure fits into the pattern
of a two eyes data set.\\
\\
Tests:
\begin{enumerate}
\item Calls 'set\_import\_values(this)' to get import data set and expected
channel data set 
\item passes returned sets to 'verify\_basic\_data\_structure()'
\end{enumerate}

\subsubsection*{test\_one\_eye}

Function name: test\_one\_eye(this)\\
Description: Test if the returned data structure fits into the pattern
of a one eye data set.\\
\\
Tests:
\begin{enumerate}
\item Creates an import data set and the expected channel data set an pass
it to 'verify\_basic\_data\_structure()'
\end{enumerate}

\subsubsection*{test\_track\_dist}

Function name: test\_track\_dist(this)\\
Description: Test if the returned data structure fits into the pattern
of a two eyes data with eyelink\_trackdist set.\\
\\
Tests:
\begin{enumerate}
\item Calls 'set\_import\_values(this)' to get import data set and expected
channel data set 
\item owerwrites some import data and channel data
\item pass returned sets to 'verify\_basic\_data\_structure()'
\end{enumerate}

\subsection{Testcases: pspm\_get\_hb}

\subsubsection{Information}

Testclass: pspm\_get\_hb\_test \\
Function: {[}sts, data{]} = pspm\_get\_hb(import) 

\subsubsection{Testcases}

\subsubsection*{Test}

Function name: test(this) \\
Description: Test if all fields are returned correctly \\
\\
Tests:
\begin{enumerate}
\item Test if \textquoteleft sts\textquoteright{} is equal 1.
\item Test if data.data is equal import.data
\item Test if data.header.chantype is \textquoteleft hb\textquoteright{}
\item Test if data.header.units is \textquoteleft events\textquoteright{}
\item Test if data.header.sr is 1 
\end{enumerate}

\subsection{Testcases: pspm\_get\_hr}

\subsubsection{Information}

Testclass: pspm\_get\_hr\_test \\
Function: {[}sts, data{]} = pspm\_get\_hr(import) 

\subsubsection{Testcases}

\subsubsection*{Test}

Function name: test(this) \\
Description: Test if all fields are returned correctly \\
\\
Tests:
\begin{enumerate}
\item Test if \textquoteleft sts\textquoteright{} is equal 1.
\item Test if data.data is equal import.data
\item Test if data.header.chantype is \textquoteleft hr\textquoteright{}
\item Test if data.header.units is equal import.units
\item Test if data.header.sr is equal import.sr 
\end{enumerate}

\subsection{Testcases: pspm\_get\_marker}

\subsubsection{Information}

Testclass: pspm\_get\_marker\_test \\
Function: {[}sts, data{]} = pspm\_get\_marker(import) 

\subsubsection{Testcases}

\subsubsection*{Test}

Function name: test(this) \\
Description: Test if all fields are returned correctly \\
\\
Tests:
\begin{enumerate}
\item Test if \textquoteleft sts\textquoteright{} is equal 1.
\item Test if data.data is equal import.data
\item Test if data.header.chantype is \textquoteleft marker\textquoteright{}
\item Test if data.header.units is \textquoteleft events\textquoteright{}
\item Test if data.header.sr is 1 
\end{enumerate}

\subsection{Testcases: pspm\_get\_pupil}

\subsubsection{Information}

Testclass: pspm\_get\_pupil\_test \\
Function: {[}sts, data{]} = pspm\_get\_pupil(import) 

\subsubsection{Testcases}

\subsubsection*{Test}

Function name: test(this) \\
Description: Test if all fields are returned correctly \\
\\
Tests:
\begin{enumerate}
\item Test if \textquoteleft sts\textquoteright{} is equal 1.
\item Test if data.data is equal import.data
\item Test if data.header.chantype is \textquoteleft pupil\textquoteright{}
\item Test if data.header.units is equal import.units
\item Test if data.header.sr is equal import.sr 
\end{enumerate}

\subsection{Testcases: pspm\_get\_resp}

\subsubsection{Information}

Testclass: pspm\_get\_resp\_test \\
Function: {[}sts, data{]} = pspm\_get\_resp(import) 

\subsubsection{Testcases}

\subsubsection*{Test}

Function name: test(this) \\
Description: Test if all fields are returned correctly \\
\\
Tests:
\begin{enumerate}
\item Test if \textquoteleft sts\textquoteright{} is equal 1.
\item Test if data.data is equal import.data
\item Test if data.header.chantype is \textquoteleft resp\textquoteright{}
\item Test if data.header.units is equal import.units
\item Test if data.header.sr is equal import.sr 
\end{enumerate}

\subsection{Testcases: pspm\_get\_scr}

\subsubsection{Information}

Testclass: pspm\_get\_scr\_test \\
Function: {[}sts, data{]} = pspm\_get\_scr(import) 

\subsubsection{Testcases}

There are three test functions. One for the case that no transfer
parameters are defined, one for the case that the transfer parameters
are defined in a struct and one for the case that they are defined
in a .mat file. They are all performing the following tests, plus
eventually some individual tests\\
\\
Tests:
\begin{enumerate}
\item Test if \textquoteleft sts\textquoteright{} is equal 1.
\item Test if the field data.data exists
\item Test if the field data.data is not empty
\item Test if the field data.header.units exists
\item Test if the field data.header.sr exists
\item Test if the field data.header.chantype exists
\item Test if data.header.sr is equal import.sr
\item Test if data.header.chantype is \textquoteleft scr\textquoteright{} 
\end{enumerate}

\subsubsection*{No transfer parameters}

Function name: no\_transferparams(testCase) \\
Description: Test if all fields are returned correctly, if no transfer
parameters are defined. \\
\\
Additional Tests: \\
\\
No additional tests 

\subsubsection*{Struct transfer parameters}

Function name: stuct\_transferparams(testCase) \\
Description: Test if all fields are returned correctly, if the transfer
parameters are defined in a struct. \\
\\
Additional Tests:
\begin{enumerate}
\item Check for warning if the conversion constant (import.transfer.c) is
not defined
\item Checks that there are no warnings if import.transfer.Rs or import.transfer.offset
is not defined. 
\end{enumerate}

\subsubsection*{File transfer parameters}

Function name: file\_transferparams(testCase) \\
Description: Test if all fields are returned correctly, if the transfer
parameters are defined in a .mat file. \\
\\
Additional Tests:
\begin{enumerate}
\item Check for warning if the transfer parameter file doesn\textquoteright t
exist. 
\end{enumerate}

\subsection{Testcases: pspm\_get\_timing}

\subsubsection{Information}

Testclass: pspm\_get\_timing\_test \\
Function: \\
{[}sts, multi{]} = pspm\_get\_timing('onsets', intiming, timeunits)
\\
{[}sts, events{]} = pspm\_get\_timing('markervalues', markerinfo,
names)\\
{[}sts, epochs{]} = pspm\_get\_timing('epochs', epochs) \\
{[}sts, events{]} = pspm\_get\_timing('events', events)\\


\subsubsection{Testcases}

\subsubsection*{Invalid input arguments}

Function name: invalid\_inputargs(this) \\
Description: Checks for warnings, if the input arguments are invalid.
\\
\\
Tests: 

\begin{tabular}{|>{\centering}p{6cm}|c|}
\hline 
Input & Expected warning\tabularnewline
\hline 
\hline 
pspm\_get\_timing('epochs') {[}missing input var{]} & ID:invalid\_input \tabularnewline
\hline 
pspm\_get\_timing('onsets', 'str') {[}no timeunits var{]} & ID:invalid\_input\tabularnewline
\hline 
pspm\_get\_timing('foo') {[}unknown format{]} & ID:invalid\_input\tabularnewline
\hline 
pspm\_get\_timing('onsets', intiming, 'samples') {[}two sessions with
nonmatching number of conditions{]} & ID:number\_of\_elements\_dont\_match\tabularnewline
\hline 
pspm\_get\_timing('onsets', intiming, 'samples') {[}two sessions with
nonmatching condition names{]} & ID:event\_names\_dont\_match\tabularnewline
\hline 
pspm\_get\_timing('onsets', intiming, 'samples') {[}intiming.onsets\{1\}
is no numeric vector{]} & ID:no\_numeric\_vector\tabularnewline
\hline 
pspm\_get\_timing('epochs', fn\_mat, 'samples') {[}epochs is not an
integer array{]} & ID:no\_integers\tabularnewline
\hline 
pspm\_get\_timing('markervalues', markerinfo) {[}no markervalue and
no name {]} & ID:invalid\_input\tabularnewline
\hline 
pspm\_get\_timing('markervalues', markerinfo, markervalue, names)
{[}markervalue is not of numeric type nor a cell array{]}  & ID:invalid\_input\tabularnewline
\hline 
pspm\_get\_timing('markervalues', markerinfo, markervalue, names)
{[}markervalue and names are not of the same length{]} & ID:invalid\_input\tabularnewline
\hline 
\end{tabular}

\subsubsection*{Case Epochs}

Function name: case\_epochs(this) \\
Description: Checks the function in \textquoteleft epochs\textquoteright{}
mode. \\
Function: {[}sts, epochs{]} = pspm\_get\_timing('epochs', epochs) 

\subsubsection*{Test 1 (matfile input)}

Input: mat file with variable: epochs = {[}1 4; 2 5; 3 6{]} \\
\\
Check if sts==1 and if the return value is equal the input array. 

\subsubsection*{Test 2 (spm input)}

Input: mat file with variable: onsets\{1\} = {[}1 2 3{]}';onsets\{2\}
= {[}4 5 6{]}'; \\
\\
Check if sts==1 and if the return value is equal {[}onsets\{1\}, onsets\{2\}{]}. 

\subsubsection*{Test 3 (textfile input)}

Input: textfile with variable: epochs = {[}1 4; 2 5; 3 6{]} \\
\\
Check if sts==1 and if the return value is equal the input array. 

\subsubsection*{Test 4 (matrix input)}

Input: matrix: epochs = {[}1 4; 2 5; 3 6{]} \\
\\
Check if sts==1 and if the return value is equal the input array. 

\subsubsection*{Case onsets}

Function name: case\_onsets(this) \\
Description: Checks the function in \textquoteleft onsets\textquoteright{}
mode. \\
Function: {[}sts, multi{]} = pspm\_get\_timing('onsets', intiming,
timeunits) 

\subsubsection*{Test 1}

Input: mat file with the variables: \\
names = \{'name1', 'name2'\}; \\
onsets = \{{[}1 2{]}, {[}3 4{]}\}; \\
pmod.name = \{'name3', 'name4'\}; \\
pmod.param = \{{[}2 3{]}, {[}4 5{]}\}; \\
pmod.poly = \{2, 2\}; \\
save(fn\_mat, 'names', 'onsets', 'pmod'); \\
\\
Function call: \\
{[}sts, outtiming{]} = pspm\_get\_timing('onsets', fn\_mat, 'samples');\\
\\
Tests: \\
Check if sts==1, if onsets and names are unchanged and if \\
outtiming.pmod.param == \{{[}2 3{]}, {[}4 9{]}, {[}4 5{]}, {[}16 25{]}\} 

\subsubsection*{Test 2}

Input: \\
mat file with the variables: names = \{'name1', 'name2'\}; \\
onsets = \{{[}1 2 3{]}, {[}3 4 5{]}\}; durations = \{{[}3 4 5{]}',
{[}5 6 7{]}'\}; \\
pmod.name = \{'name3', 'name4'\}; \\
pmod.param = \{{[}2 3 4{]}, {[}4 5 6{]}\}; \\
pmod.poly = \{2, 1\};\\
\\
Function call: \\
{[}sts, outtiming{]} = pspm\_get\_timing('onsets', fn\_mat, 'samples');\\
\\
Tests: \\
Check if sts==1, if onsets,names and durations are unchanged and if
\\
outtiming.pmod.param == \{{[}2 3 4{]}, {[}4 9 16{]}, {[}4 5 6{]}\} 

\subsubsection*{Case events}

Function name: case\_events(this) \\
Description: Checks the function in \textquoteleft events\textquoteright{}
mode. \\
Function: {[}sts, epochs{]} = pspm\_get\_timing('events', events)
\\
\\
Check the function if input is a one element cell array and a multiple
element cell array. Check for warnings (ID:invalid\_vector\_size)
if elements have more than two columns and if not all elements have
the same number of rows.

\subsection{Testcases: pspm\_get\_<datatype>}

\subsubsection{Information}

The datatype import functions are all tested in a similar way. The
individual testclasses must inherit the class \textquoteleft pspm\_get\_superclass\textquoteright ,
from which they inherit the main test function \textquoteleft valid\_datafile\textquoteright .
They also have to implement the property \textquoteleft fhandle\textquoteright ,
which is a function handle to the specific import function. \\
\\
The tests are performed with the sampledata files that are listed
in the SampleDataMasterList.docx file (as at 18.11.2013). \\
\\
Superclass: pspm\_get\_superclass\\
\\
Testclasses: pspm\_get\_acq\_test \\
pspm\_get\_acqmat\_test \\
pspm\_get\_biograph\_test \\
pspm\_get\_biosemi\_test \\
pspm\_get\_biotrace\_test \\
pspm\_get\_brainvis\_test\\
pspm\_get\_edf\\
pspm\_get\_labchartmat\_ext\_test \\
pspm\_get\_labchartmat\_in\_test \\
pspm\_get\_mat\_test \\
pspm\_get\_obs\_test \\
pspm\_get\_spike\_test \\
pspm\_get\_superclass \\
pspm\_get\_txt\_test \\
pspm\_get\_vario\_test\\
pspm\_get\_eyelink\_test\\
\\
Function: {[}sts, import, sourceinfo{]} = pspm\_get\_<datatype>(datafile,
import) 

\subsubsection{Notes}

\subsubsection{Setup}

\subsubsection*{define testcases}

In this method the testcases are defined and the testdata is generated
(if needed). Each testcase is a cell in the cellarray \textquoteleft testcases\textquoteright .
Each testcase has the following fields:
\begin{itemize}
\item .pth: the path to the samplefile
\item .import: the input variable
\end{itemize}
For datatypes which support blocks there has to be an additional field:
\begin{itemize}
\item .numofblocks 
\end{itemize}

\subsubsection{Testcases}

\subsubsection*{Valid datafile}

Function name: valid\_datafile(this) \\
\\
Description: The main test function, for tests with valid inputdata.
It tests all testcases equally. \\
\\
Tests:
\begin{enumerate}
\item Test if \textquoteleft sts\textquoteright{} is equal 1.
\item If the datatype supports blocks, test if the number of blocks is correct.
\item Test if number of elements of the returned \textquoteleft import\textquoteright{}
variable is correct.
\item Test if each importjob has a field \textquoteleft data\textquoteright ,
that is a numeric vector.
\item Test if each importjob has a field \textquoteleft sr\textquoteright ,
that is a number.
\item Test if each importjob has a field \textquoteleft type\textquoteright .
\item Test if all event importjobs have a field \textquoteleft marker\textquoteright .
\item Test if all importjobs have duration below 1h.
\item Test if all importjobs have a samplerate between 1 and 10000 for continuous
channels or between 10\textasciicircum -6 and 1 for timestamp channels. 
\end{enumerate}

\subsubsection*{invalid datafile}

Function name: invalid\_datafile(this) \\
Description: The main test function, for tests with invalid inputdata.
\\
\\
Tests: \\
\\
If the datatype supports multiple channels: Check for warning when
trying to import a channel, that is not contained in the file ('ID:channel\_not\_contained\_in\_file'). 

\subsection{Testcases: pspm\_get\_acq}

In this section we describe the testcases specific to pspm\_get\_acq
apart from generic pspm\_get tests.

\subsubsection{Information}

Testclass: pspm\_get\_acq\_test\\
Function: {[}tss, import, sourceinfo{]} = pspm\_get\_acq(datafile,
import)

\subsubsection{Testcases}

\subsubsection*{get\_acq should return the exact same data as Acqknowledge exported
mat file}

Function name: get\_acq\_returns\_same\_data\_as\_acqknowledge\_exported\_mat(this)\\
Description: The data obtained by using pspm\_get\_acq should be identical
with the data obtained by using export .mat file functionality in
Acqknowledge software.\\
\\
Tests:
\begin{enumerate}
\item Load data stored in \texttt{ImportTestData/acq/impedance\_acq.acq}
and \texttt{ImportTestData/acq/impedance\_mat.mat} separately. Then
compare the first channel of \texttt{impedance\_mat} with the first
channel of data obtained by calling \texttt{pspm\_get\_acq} on \texttt{impedance\_acq}.
\end{enumerate}

\subsection{Testcases: pspm\_glm}

\subsubsection{Information}

Testclass: pspm\_glm\_test \\
Function: glm = pspm\_glm(model, options) \\
\\
There are seven testcase functions. One invalid input arguments test
and test 1 to 6. Tests 1 to 5 are of the same kind. There are one
or multiple testcases per test function, have a look at the testcase
description for more information. In these tests only Kronecker delta
functions are used as basis functions, furthermore all conditions,
pmods and nuisance regressors are pairwise orthogonal. The data is
also not down sampled and not filtered. With these limitations it\textquoteright s
easy to calculate the data vectors and the expected stats. For each
testcase it is then tested:
\begin{itemize}
\item If numel(glm.names) has the expected value.
\item If numel(glm.stats) has the expected value.
\item If glm.stats has the expected value (with a tolerance of 1\%).
\end{itemize}
In test 6 the default basis functions are used, and not all conditions
and pmods are orthogonal. The data is down sampled and low and high
pass filtered. In exchange the stats are not tested for correct values,
just for the correct number of elements. The properties 'shiftbf'
and 'norm' are TestParameters, which means that this testclass is
parameterized. All functions implmementing these parameters (Test
1 to Test 5) are called several times with all the different values
and combinations of the mentioned parameters.

\subsubsection{Testcases}

\subsubsection*{Invalid input arguments}

Function name: invalid\_input (this) \\
Description: Checks for warnings, if the input arguments are invalid.
\\
\\
Tests:\\
\\
\begin{tabular}{|>{\centering}p{6cm}|c|}
\hline 
Input & Expected warning\tabularnewline
\hline 
\hline 
pspm\_glm(model) {[}no timeunits field{]} & ID:invalid\_input\tabularnewline
\hline 
pspm\_glm(model) {[}no timeunits var{]} & ID:invalid\_input\tabularnewline
\hline 
pspm\_glm(model) with model.timeunits = \textquoteleft foo\textquoteright{}
{[}no valid timeunits field{]} & ID:invalid\_input\tabularnewline
\hline 
pspm\_glm(model) with model.timing = zeros(10,2) {[}no valid timing
field{]} & ID:invalid\_input\tabularnewline
\hline 
pspm\_glm(model) with model.modality = 'foo' {[}no valid modality
field{]} & ID:invalid\_input\tabularnewline
\hline 
pspm\_glm(model) with model.channel = 'scr' {[}no valid channel field{]} & ID:invalid\_input\tabularnewline
\hline 
pspm\_glm(model) with model.norm = 'no' {[}no valid norm field{]} & ID:invalid\_input\tabularnewline
\hline 
pspm\_glm(model) with model.filt.down = 'none' {[}filt.down is not
numeric{]} & ID:invalid\_input\tabularnewline
\hline 
pspm\_glm(model) with model.bf.fhandle = 'foohandle' {[}non existing
bf{]} & ID:invalid\_fhandle\tabularnewline
\hline 
pspm\_glm(model) with numel(model.datafile) != numel(model.timing) & ID:number\_of\_elements\_dont\_match\tabularnewline
\hline 
pspm\_glm(model) with model.missing is struct {[}non valid missing
field{]} & ID:invalid\_input\tabularnewline
\hline 
pspm\_glm(model) with numel(model.datafile) != numel(model.missing) & ID:number\_of\_elements\_dont\_match\tabularnewline
\hline 
pspm\_glm(model) with model.nuisance is struct {[}non valid nuisance
field{]} & ID:invalid\_input\tabularnewline
\hline 
pspm\_glm(model) with numel(model.datafile) != numel(model.nuisance) & ID:number\_of\_elements\_dont\_match\tabularnewline
\hline 
pspm\_glm(model) with no R variable in the nuisance file & ID:invalid\_input\tabularnewline
\hline 
\multicolumn{1}{|>{\centering}p{4cm}|}{pspm\_glm(model) with R variable in the nuisance file that has not
the same length as the datafile} & ID:number\_of\_elements\_dont\_match\tabularnewline
\hline 
\end{tabular}

\subsubsection*{Test 1}

Function name: test1(this, shiftbf) \\
Description: Basic test with one basis function, one session, no nuisance
regressors, no missings and one condition. Timeunits are seconds.
\\
\\
Testcases:
\begin{enumerate}
\item no pmods
\item one pmod
\item two pmods 
\end{enumerate}

\subsubsection*{Test 2}

Function name: test2(this, shiftbf) \\
Description: Test with one basis function, one session, no nuisance
regressors, no missings and two conditions. Timeunits are seconds.
\\
\\
Testcases:
\begin{enumerate}
\item no pmods
\item first condition: no pmods; second condition: one pmod
\item first condition: one pmod; second condition: two pmods 
\end{enumerate}

\subsubsection*{Test 3}

Function name: test3(this, shiftbf) \\
Description: Test with one basis function, one session, two nuisance
regressors (1Hz cosinus, 1Hz sinus), no missings, one condition and
no pmods. Timeunits are seconds. \\
\\
Testcases: \\
\\
Only one testcase. 

\subsubsection*{Test 4}

Function name: test4(this, shiftbf) \\
Description: Test with one basis function, two sessions, no nuisance
regressors, no missings and one condition. \\
\\
Testcases:
\begin{enumerate}
\item timeunits are seconds
\item timeunits are samples
\item timeunits are markers 
\end{enumerate}

\subsubsection*{Test 5}

Function name: test5(this, shiftbf) \\
Description: Test with two basis functions, one session, no nuisance
regressors and one condition. Timeunits are seconds. \\
\\
Testcases:
\begin{enumerate}
\item no missings
\item with missings 
\end{enumerate}

\subsubsection*{Test 6}

Function name: test6(this) \\
Description: Test with default basis function and non-orthogonal conditions
and pmods \\
\\
Testcase: \\
\\
Default basis functions, no nuisance regressors, no missings, two
sessions and two conditions. Timeunits are seconds.
\begin{itemize}
\item first condition: two pmods (with pmod(1).poly\{1\} = 2 and pmod(1).poly\{2\}
= 3)
\item second condition: no pmods 
\end{itemize}

\subsubsection*{Test7:}

Function name: test\_extract\_missing(this, cutoff, nan\_percent)\\
Description: Test with one basis function, one session, no nuisance
regressors, no missings and three conditions. Timeunits are seconds.
\\
\\
Testcases: for all combinations of the test parameters cutoff and
nan\_percent
\begin{itemize}
\item glm vector stats\_missing has the appropriate length according to
the number of conditions
\item percentages in glm vector stats\_missing contains the expected value
\item glm vector stats\_exclude has the appropriate length according to
the number of conditions
\item glm vector stats\_exclude contains the expected condistion which should
be excluded 
\end{itemize}

\subsection{Testcases: pspm\_hb2hp}

\subsubsection{Information}

Testclass: pspm\_hb2hp\_test \\
Function: {[}sts, infos{]} = pspm\_hb2hp(fn, sr, chan, options)

\subsubsection{Testcases}

\subsubsection*{Invalid input}

Function name: invalid\_input(this) \\
Description: Checks for warnings, if the input arguments are invalid.
\\
\\
Tests: 

\begin{tabular}{|c|c|}
\hline 
Input & Expected warning\tabularnewline
\hline 
\hline 
pspm\_hb2hp() {[}no input{]} & ID:invalid\_input\tabularnewline
\hline 
pspm\_hb2hp(2) {[}not a string filename{]} & ID:invalid\_input\tabularnewline
\hline 
pspm\_hb2hp('abc') {[}no sample rate{]} & ID:invalid\_input\tabularnewline
\hline 
pspm\_hb2hp('abc','abc') {[}not a string sample rate{]} & ID:invalid\_input\tabularnewline
\hline 
pspm\_hb2hp('abc',2,'abc') {[}not a numeric chanel{]} & ID:invalid\_input\tabularnewline
\hline 
pspm\_hb2hp(files\{1\},100) {[}call of pspm\_load\_data fails{]} & ID:invalid\_input\tabularnewline
\hline 
pspm\_hb2hp(files\{2\}, 100) {[}not enough points for interp1{]} & ID:too\_strict\_limits\tabularnewline
\hline 
pspm\_hb2hp(files\{3\},100,{[}{]},options) {[}pspm\_write\_channel
fails{]} & ID:invalid\_input\tabularnewline
\hline 
\end{tabular}

\subsection{Testcases: pspm\_import}

\subsubsection{Information}

Testclass: pspm\_import\_test \\
Function: outfile = pspm\_import(datafile, datatype, import, options) 

\subsubsection{Testcases}

\subsubsection*{Invalid input arguments}

Function name: invalid\_inputargs(ths) \\
Description: Checks for warnings, if the input arguments are invalid.
\\
\\
Tests: \\
\\
\begin{tabular}{|c|>{\centering}p{6cm}|c|}
\hline 
Test No. & Input & Expected warning\tabularnewline
\hline 
\hline 
1 & pspm\_import(datafile, datatype) {[}no import variable{]} & ID:invalid\_input \tabularnewline
\hline 
2 & pspm\_import(datafile, datatype, \textquoteleft foo\textquoteright )
{[}no cell/struct import var.{]} & ID:invalid\_input\tabularnewline
\hline 
3 & pspm\_import(datafile, \textquoteleft foo\textquoteright , import)
{[}invalid channeltype{]} & ID:invalid\_channeltype\tabularnewline
\hline 
4 & pspm\_import(5, datatype, import) {[}no char filename{]} & ID:invalid\_input\tabularnewline
\hline 
\end{tabular}

\subsubsection*{Invalid import variable structure}

Function name: invalid\_import\_struct(this) \\
Description: Checks for warnings, if the structure of the import variable
is invalid. \\
\\
Tests: \\
\\
\begin{tabular}{|c|c|c|}
\hline 
Test No. & Input & Expected warning\tabularnewline
\hline 
\hline 
1 & Multiple channel, though not supported & ID:invalid\_import\_struct\tabularnewline
\hline 
2 & Not allowed channeltype & ID:invalid\_import\_struct\tabularnewline
\hline 
3 & No sr given, though autosr is not supported & ID:invalid\_import\_struct\tabularnewline
\hline 
4 & Nonexistent file & ID:nonexistent\_file\tabularnewline
\hline 
\end{tabular}

\subsubsection*{One datafile}

Function name: one\_datafile(this) \\
Description: Checks the function, if datafile is a string (import
of one datafile) and all inputs are correct. The outfile is checked
with the pspm\_load\_data function. The tests are performed with a
spike samplefile and a labchartmat\_in samplefile (to check the handling
of blocks). 

\subsubsection*{Multiple datafiles}

Function name: multiple\_datafiles(this) \\
Description: Checks the function, if datafile is a cell array of strings
(import of multiple datafiles) and all inputs are correct. The outfiles
are tested with the pspm\_load\_data function. 

\subsection{Testcases: pspm\_interpolate}

\subsubsection{Information}

Testclass: pspm\_interpolate\_test\\
Function: {[}sts, outdata{]} = pspm\_interpolate(indata, options)

\subsubsection{Setup}

This test class is parameterized. The test data is generated by the
function itself and when needed, files will be written to datafile<variable\_nr>.mat.

\subsubsection*{Test parameters}

These are parameters which define what kind of data should be passed
to pspm\_interpolate and which options should be set.\\
\begin{tabular}{|c|>{\raggedright}p{10cm}|}
\hline 
\textbf{Amount} & Specifies how many elements indata (for pspm\_interpolate) should
have.\tabularnewline
\hline 
\textbf{Datatype} & Specifies what type of data should be generated.
\begin{itemize}
\item struct - a valid data struct will be generated
\item inline - a numeric vector will be generated
\item file - a valid scr file will be generated
\item all - all types will sequentially be generated until amount is reached
\end{itemize}
\tabularnewline
\hline 
\textbf{Chans} & If datatype is not inline this specifies how many and which type of
data channels the generated data should have. In a second field it
also defines which of these channels should be interpolated (this
will be passed later in options.channels).\tabularnewline
\hline 
\textbf{Nan method} & Specifies how NaN values will be put into the data.
\begin{itemize}
\item start - range is 1+offset:<random number before the center>
\item center - range is <radnom number before the center>:<random number
after the center>
\item end - range is <random number after the center>:end-offset
\end{itemize}
The offset is 1 if 'extrap' is not defined. This is needed because
if there is no data at the end or beginning of the data, the function
is unable to interpolate (unless extrapolation is activated).\tabularnewline
\hline 
\textbf{Extrap} & Is either true or false and activates or deactivates the extrapolation.\tabularnewline
\hline 
\textbf{Interp method} & Specifies the interpolation method.\tabularnewline
\hline 
\textbf{Newfile} & True or false and tells the function to either create a file or add
the data as new channel.\tabularnewline
\hline 
\textbf{Overwrite} & True or false and tells the function to either overwrite an existing
file or not.\tabularnewline
\hline 
\textbf{Replace channel} & True or false and tells the function to either replace the given channels
with the interpolated data or to add the interpolated data as new
channel.\tabularnewline
\hline 
\end{tabular}

\subsubsection{Testcases}

\subsubsection*{Invalid input}

Function name: invalid\_input(this) \\
Description: Checks for warnings, if the input arguments are invalid.
\\
\\
Tests: \\
\\
\begin{tabular}{|c|>{\centering}p{6cm}|c|}
\hline 
Test No. & Input & Expected warning\tabularnewline
\hline 
\hline 
1 & pspm\_interpolate() {[}no arguments{]} & ID:missing\_data\tabularnewline
\hline 
2 & pspm\_interpolate(\{\{\}\}) {[}data is not char, struct, numeric{]} & ID:invalid\_input\tabularnewline
\hline 
3 & pspm\_interpolate(\{\}) {[}data empty{]} & ID:missing\_data\tabularnewline
\hline 
4 & pspm\_interpolate(struct()) {[}invalid struct{]} & ID:invalid\_data\_structure\tabularnewline
\hline 
5 & pspm\_interpolate(invalid\_data) {[}file which does not exist{]} & ID:nonexistent\_file\tabularnewline
\hline 
6 & pspm\_interpolate(valid\_data, options) {[}options.channels is larger
than valid\_data{]} & ID:invalid\_input\tabularnewline
\hline 
7 & pspm\_interpolate(valid\_data, options) {[}options.channels is not
numeric{]} & ID:invalid\_input\tabularnewline
\hline 
8 & pspm\_interpolate(valid\_data, options) {[}options.method is invalid{]} & ID:invalid\_input\tabularnewline
\hline 
9 & pspm\_interpolate(valid\_data, options) {[}options.newfile is invalid{]} & ID:invalid\_input\tabularnewline
\hline 
10 & pspm\_interpolate(valid\_data, options) {[}options.extrapolate is
invalid{]} & ID:invalid\_input\tabularnewline
\hline 
11 & pspm\_interpolate(valid\_data, options) {[}options.overwrite is invalid{]} & ID:invalid\_input\tabularnewline
\hline 
12 & pspm\_interpolate(valid\_data, options) {[}options.dont\_ask\_overwrite
is invalid{]} & ID:invalid\_input\tabularnewline
\hline 
13 & pspm\_interpolate(valid\_data, options) {[}options.replace\_channels
is invalid{]} & ID:invalid\_input\tabularnewline
\hline 
14 & pspm\_interpolate(invalid\_data, options) {[}try to interpolate an
events channel{]} & ID:invalid\_channeltype\tabularnewline
\hline 
15 & pspm\_interpolate(invalid\_data) {[}try to interpolate with nan from
beginning and without extrapolation{]} & ID:option\_disabled\tabularnewline
\hline 
16 & pspm\_interpolate(invalid\_data, options) {[}try to interpolate with
nan from beginning and with extrapolation{]} & ID:out\_of\_range\tabularnewline
\hline 
17 & pspm\_interpolate(invalid\_data) {[}try to interpolate with nan from
end and without extrapolation{]} & ID:option\_disabled\tabularnewline
\hline 
18 & pspm\_interpolate(invalid\_data, options) {[}try to interpolate with
nan from end and with extrapolation{]} & ID:out\_of\_range\tabularnewline
\hline 
\end{tabular}

\subsubsection*{Test datatypes}

Function name: test\_datatypes(this, datatype, amount, chans) \\
Description: Tries to interpolate with different datatypes, amount
of data, channels.\\
\\
Tests: 
\begin{enumerate}
\item Generate data with datatype, amount, 'center', chans, false
\item Test if function issues no warnings
\item Test if sts is 1
\item Test if size of outdata equals the size of the data
\item Test if channels to be interpolated have no more NaNs
\item Test if channels not to be interpolated still contain NaNs
\end{enumerate}

\subsubsection*{Test interpolation variations}

Function name: test\_interpolation\_variations(this, interp\_method,
extrap, nan\_method) \\
Description: Tries to interpolate with different interpolation methods
while varying options.extrapolate and the nan\_method.\\
\\
Tests: 
\begin{enumerate}
\item Generate data with 'inline', 1, nan\_method, \{\{'scr'\}, {[}{]}\},
extrap
\item Test if function issues no warnings
\item Test if sts is 1
\item Test if size of outdata equals the size of the data
\item Test if data has no more NaNs
\end{enumerate}
\textbf{\textit{Special case:}} When extrapolation is on and nan\_method
is 'start' and interp\_method is 'previous' or nan\_method is 'end'
and interp\_method is 'next'. This should issue a warning because
this is not possible (e.g. interpolate with previous value when first
NaN value is at the beginning of the data set).
\begin{enumerate}
\item Generate data as above
\item Test if function issues a warning.
\end{enumerate}

\subsubsection*{Test no nan}

Function name: test\_no\_nan(this) \\
Description: Test whether function works even if there is nothing
to interpolate.\\
\\
Tests:
\begin{enumerate}
\item Generate data struct() with pspm\_test\_data\_gen()
\item Test if function issues no warnings
\item Test if sts is 1
\item Test if size of outdata equals the size of data
\item Test if outdata equals data
\item Test if data has no NaNs
\end{enumerate}

\subsubsection*{Test write}

Function name: test\_write(this, newfile) \\
Description: Vary the option newfile and test whether new file is
created correctly or data is correctly added to a new channel.\\
\\
Tests:
\begin{enumerate}
\item Generate data with 'file', 2, 'center', \{\{'scr', 'scr', 'scr'\},
{[}1,3{]}\} , false
\item Test if function issues no warnings
\item Test if sts is 1
\item Test if size of outdata equals the size of data
\item Test if outdata does not equal data
\end{enumerate}
New files only:
\begin{enumerate}
\item Test if new file exists
\item Load old and new file and test if size of data is equal
\item Verify that interpolated channels in the new file are NaN free
\end{enumerate}
Added to existing file only:
\begin{enumerate}
\item Test if all returned values are numeric (new channel ids)
\item Verify that the added channels are NaN free
\item Test if added channels match the size of the original data channels
\end{enumerate}

\subsubsection*{Test overwrite}

Function name: test\_overwrite(this, overwrite) \\
Description: Vary overwrite and test whether files are overwritten
or not.\\
\\
Tests: 
\begin{enumerate}
\item Generate data with 'file', 2, 'center', \{\{'scr', 'scr', 'scr'\},
{[}1,2,3{]}\}, false
\item Create files with expected filenames
\item Test if function issues no warning
\item Test if sts is 1
\item According to overwrite test if file has been overwritten or not
\end{enumerate}

\subsubsection*{Test replace channel}

Function name: test\_replace\_channel(this, replace\_channels)\\
Description: Vary replace\_channel and test wether channels are overwritten
or not.\\
\\
Tests:
\begin{enumerate}
\item Generate data with 'file', 2, 'center', \{\{'scr', 'scr', 'scr'\},
{[}1,2,3{]}\} , false
\item Test if function issues no warnings
\item Test if sts is 1
\item Test if size of outdata equals the size of data
\item Test if outdata does not equal data
\item According to replace\_channel test whether returned channel ids correspond
to replaced channels or correspond to added channels.
\end{enumerate}

\subsubsection{Other methods}

\subsubsection*{Generate data}

Has all of the Test parameters as parameter implemented and accordingly
generates the data. It calls put nan to insert NaN values into the
data. The generated data is returned as data to the calling function.
Also all return values are stored in the property testdata (for cleanup
data).

\subsubsection*{Cleanup data}

Sits in MethodTeardown and is called once the test class has finished
all tests. It then removes all the datafiles which can be found in
the property 'testdata'.

\subsubsection*{Verify NaN free}

Helper function to verify whether the data is NaN free or not. It
copes with two states. Either a channel should have been interpolated,
then it shouldn't contain any NaN values or a channel should not have
been interpolated, then the channel should still contain NaN values.

\subsection{Testcases: pspm\_load1}

\subsubsection{Information}

Testclass: pspm\_load1\_test\\
Function: {[}sts, data, mdltype{]} = pspm\_load1(fn, action, savedata,
options)

\subsubsection{Setup}

The datafile fn is referring to a datafile which was generated with
pspm\_load1\_test.generate\_testdata(this). The function is part of
the test object and generates models for all of the available model
types (defined in settings.first). The models are created with data
generated with pspm\_testdata\_gen. Two files belong to each model:
model\_<modeltype><variable nr.>.mat (fn) and dummy\_<modeltype><variable\_nr>.mat
(dfn). The model file on the one hand is the actual model file while
on the other hand, the dummy file is a copy of the model file, used
by the test to manipulate the test data.

\subsubsection*{Generated aquisition data (pspm\_testdata\_gen)}

\texttt{data\{1\}.chantype = 'scr'; }~\\
\texttt{data\{2\}.chantype = 'hb';}~\\
\texttt{}~\\
The duration of the channels is 100s.

\subsubsection*{Generated GLM model}

\texttt{model.timing\{1\}.names = \{'a';'b';'c'\}; }~\\
\texttt{model.timing\{1\}.onsets = \{{[}10, 20, 30{]}, {[}15, 25,
35{]}, {[}18, 28, 38{]}\}; }

\subsubsection*{Generated DCM \& SF model}

\texttt{model.timing\{1\} = {[}10,20; 23,38; 40,70;{]}; }~\\
\texttt{model.condition\{1\}.name = \{'a';'b'\}; }~\\
\texttt{model.condition\{1\}.index = {[}1;2{]};}

\subsubsection{Testcases}

\subsubsection*{Invalid model structure (general)}

Function: invalid\_model\_tructure\_general(this)\\
Description: Tries to pass invalid data structures, and tests for
certain warnings. Applys to all available modeltypes.\\
\\
Tests: \\
\\
\begin{tabular}{|c|c|}
\hline 
Input & Expected warning\tabularnewline
\hline 
\hline 
empty model file & ID:invalid\_data\_structure\tabularnewline
\hline 
missing field 'modelfile' & ID:invalid\_data\_structure\tabularnewline
\hline 
missing field 'modeltype' & ID:invalid\_data\_structure\tabularnewline
\hline 
missing field 'modality' & ID:invalid\_data\_structure\tabularnewline
\hline 
missing field 'stats' & ID:invalid\_data\_structure\tabularnewline
\hline 
missing field 'names' & ID:invalid\_data\_structure\tabularnewline
\hline 
\end{tabular}

\subsubsection*{Invalid model structure (specific)}

Function: invalid\_model\_structure\_general(this)\\
Description: Tries to pass invalid data structures, and tests for
certain warnings. Model specific.\\
\\
Tests for GLM:\\
\\
\begin{tabular}{|c|c|}
\hline 
Input & Expected warning\tabularnewline
\hline 
\hline 
field 'stats' is not an n x 1 vector & ID:invalid\_data\_structure\tabularnewline
\hline 
unequal amount of numbers and parameters in field 'stats' & ID:invalid\_data\_structure\tabularnewline
\hline 
options.zscored = 1 \& action = 'cond' & ID:invalid\_input\tabularnewline
\hline 
\end{tabular}\\
\\
Tests for DCM \& SF:\\
\\
\begin{tabular}{|c|c|}
\hline 
Input & Expected warning\tabularnewline
\hline 
\hline 
unequal size for fields in 'trlnames' and rows in 'stats' & ID:invalid\_data\_structure\tabularnewline
\hline 
missing field 'trlnames' & ID:invalid\_data\_structure\tabularnewline
\hline 
unequal size for fields in 'names' and columns in 'stats' & ID:invalid\_data\_structure\tabularnewline
\hline 
action = 'recon' & ID:invalid\_input\tabularnewline
\hline 
\end{tabular}\\
\\
Tests for DCM:\\
\\
\begin{tabular}{|c|c|}
\hline 
Input & Expected warning\tabularnewline
\hline 
\hline 
options.zscored = 1 \& pspm\_load1(dfn, 'none', \{\}, options) & ID:invalid\_input\tabularnewline
\hline 
options.zscored = 1 \& pspm\_load1(dfn, 'cond', \{\}, options) & -\tabularnewline
\hline 
options.zscored = 1 \& pspm\_load1(dfn, 'stats', \{\}, options) & -\tabularnewline
\hline 
\end{tabular}\\
\\
Tests for GLM \& SF:\\
\\
\begin{tabular}{|c|c|}
\hline 
Input & Expected warning\tabularnewline
\hline 
\hline 
options.zscored = 1 \& pspm\_load1(dfn, 'cond', \{\}, options) & ID:invalid\_input\tabularnewline
\hline 
\end{tabular}

\subsubsection*{Action 'none'}

Function: test\_action\_none(this)\\
Description: Test for all modeltypes if action 'none' matches the
expected behaviour.\\
\\
Tests:
\begin{enumerate}
\item Basic function test
\item Test if returned data is empty.
\end{enumerate}

\subsubsection*{Action 'stats'}

Function: test\_action\_stats(this)\\
Description: Test for all modeltypes if action 'stats' matches the
expected behaviour.\\
\\
Tests for all:
\begin{enumerate}
\item Basic function test
\item Returned data contains field 'stats'
\item Returned data contains field 'names'
\end{enumerate}
Tests for DCM \& SF:
\begin{enumerate}
\item Returned data contains field 'trlnames'
\item Returned data contains field 'condnames'
\end{enumerate}

\subsubsection*{Action 'cond'}

Function: test\_action\_cond(this)\\
Description: Test for all modeltypes if action 'cond' matches the
expected behaviour.\\
\\
Tests for all:
\begin{enumerate}
\item Basic function test
\item Returned data contains field 'stats'
\item Returned data contains field 'names'
\end{enumerate}
Tests for DCM \& SF:
\begin{enumerate}
\item Returned data contains field 'trlnames'
\item Returned data contains field 'condnames'
\end{enumerate}

\subsubsection*{Action 'recon'}

Function: test\_action\_recon(this)\\
Description: Test for all modeltypes if action 'recon' matches the
expected behaviour.\\
\\
Tests for GLM:
\begin{enumerate}
\item Basic function test
\item Returned data contains field 'stats'
\item Returned data contains field 'names'
\end{enumerate}
Tests for DCM \& SF already done in specific structure test.

\subsubsection*{Action 'savecon'}

Function: test\_action\_savecon(this)\\
Description: Test for all modeltypes if action 'savecon' matches the
expected behaviour. Generates a number, passes it within the 'savecon'
struct and tests if the number is returned correctly.\\
\\
Tests:
\begin{enumerate}
\item Basic function test
\item Returned data contains field 'con'
\item Field 'con' contains field 'test'
\item Field 'con.test' is equal to the randomly generated number
\end{enumerate}

\subsubsection*{Action 'con'}

Function: test\_action\_con(this)\\
Description: Test for all modeltypes if action 'con' matches the expected
behaviour. Tests if the in 'savecon' generated field test is still
returned.\\
\\
Tests:
\begin{enumerate}
\item Basic function test
\item Returned data contains field 'con'
\item Field 'con' contains field 'test'.
\end{enumerate}

\subsubsection*{Action 'all'}

Function: test\_action\_all(this)\\
Description: Test for all modeltypes if action 'all' matches the expected
behaviour.\\
\\
Tests:
\begin{enumerate}
\item Basic function test
\item Returned data is not empty.
\end{enumerate}

\subsubsection*{Action 'save'}

Function: test\_action\_save(this)\\
Description: Test for all modeltypes if action 'save' matches the
expected behaviour. Test with options.overwrite = 1. Generates random
number and writes it into field 'test' in model structure. \\
\\
Tests:
\begin{enumerate}
\item Basic function test
\item Model structure contains field 'test'
\item Field 'test' in model structure equals to the randomly generated number.
\end{enumerate}

\subsubsection*{Options}

Function: test\_options(this)\\
Description: Test for all modeltypes if options passed with options
structure cause the expected behaviour. Does also work with a randomly
generated number in <model struct>.test to test whether the data is
written or not. \\
\\
Tests for all: 
\begin{enumerate}
\item dont\_ask\_overwrite = 1 \& overwrite = 0 returns warning ID:not\_saving\_data
and field 'test' in model struct does not match generated number
\item dont\_ask\_overwrite = 1 \& overwrite = 1 field 'test' in returned
model struct does match generated number
\end{enumerate}
Tests for DCM (with dont\_ask\_overwrite = 1 \& overwrite = 1):
\begin{enumerate}
\item zscored = 0 \& action = 'stats'

\begin{enumerate}
\item Basic function test
\item Returned data.stats is not zscored
\end{enumerate}
\item zscored = 1 \& action = 'stats'

\begin{enumerate}
\item Basic function test
\item Returned data.stats is zscored
\end{enumerate}
\item zscored = 0 \& action = 'cond'

\begin{enumerate}
\item Basic function test
\item Returned data is different when callng with zscroed = 1 \& action
= 'cond' (should not zscore, when not specified)
\end{enumerate}
\end{enumerate}

\subsubsection{Other methods}

\subsubsection*{Remove testdata}

Removes all the test data generated by the test class. It is called
once the class is finished with testing. 

\subsubsection*{Basic function test}

Is called in each test after the tested function has been called.
It does two checks:
\begin{itemize}
\item Returned modeltype matches the modeltype stored in the returned model
structure
\item Returned status (sts) equals 1
\end{itemize}

\subsection{Testcases: pspm\_load\_data}

\subsubsection{Information}

Testclass: pspm\_load\_data\_test \\
Function: {[}sts, infos, data, filestruct{]} = pspm\_load\_data(fn,
chan) 

\subsubsection{Setup}

If not otherwise declared, the input variable fn is referring to a
datafile which was generated with pspm\_testdata\_gen and consists
out of the following channels: \\
\\
\texttt{data\{1\}.chantype = 'scr'; }~\\
\texttt{data\{2\}.chantype = 'marker'; }~\\
\texttt{data\{3\}.chantype = 'hr'; }~\\
\texttt{data\{4\}.chantype = 'hb'; }~\\
\texttt{data\{5\}.chantype = 'marker'; }~\\
\texttt{data\{6\}.chantype = 'resp'; }~\\
\texttt{data\{7\}.chantype = 'scr';}\\
\\
The duration of the channels is 10s. 

\subsubsection{Testcases}

\subsubsection*{Invalid input arguments}

Function name: invalid\_inputargs(testCase) \\
Description: Checks for warnings, if the input arguments are invalid.
\\
\\
Tests: \\
\\
\begin{tabular}{|c|c|}
\hline 
Input & Expected warning\tabularnewline
\hline 
\hline 
pspm\_load\_data {[}no filename{]} & ID:invalid\_input\tabularnewline
\hline 
pspm\_load\_data(1) {[}no char filename{]} & ID:invalid\_input\tabularnewline
\hline 
pspm\_load\_data(fn, -1) {[}neg. channel no{]} & ID:invalid\_input\tabularnewline
\hline 
pspm\_load\_data(fn, \textquoteleft foobar\textquoteright ) {[}no
allowed ch type{]} & ID:invalid\_input\tabularnewline
\hline 
pspm\_load\_data(fn, foo) {[}missing field in foo struct{]} & ID:invalid\_input\tabularnewline
\hline 
pspm\_load\_data(fn, \{1\}) {[}invalid channel option{]} & ID:invalid\_input\tabularnewline
\hline 
pspm\_load\_data(struct) {[}struct has no infos field{]} & ID:invalid\_input\tabularnewline
\hline 
pspm\_load\_data(fn, 250) {[}nonexisting channel{]} & ID:invalid\_input\tabularnewline
\hline 
\end{tabular}

\subsubsection*{Invalid datafile}

Function name: invalid\_datafile(testCase) \\
Description: Checks for warnings, if the datafile is invalid. \\
\\
Tests: \\
\\
\begin{tabular}{|c|>{\centering}p{6cm}|c|}
\hline 
Test No. & Input & Expected warning\tabularnewline
\hline 
\hline 
1 & non-existent datafile & ID:nonexistent\_file\tabularnewline
\hline 
2 & missing \textquoteleft infos\textquoteright{} variable & ID:invalid\_data\_structure\tabularnewline
\hline 
3 & missing \textquoteleft data\textquoteright{} variable & ID:invalid\_data\_structure\tabularnewline
\hline 
4 & missing \textquoteleft data\textquoteright{} field in \textquoteleft data\{2\}\textquoteright{} & ID:invalid\_data\_structure\tabularnewline
\hline 
5 & missing \textquoteleft header\textquoteright{} field \textquoteleft data\{3\}\textquoteright{} & ID:invalid\_data\_structure\tabularnewline
\hline 
6 & missing \textquoteleft sr\textquoteright{} field in \textquoteleft data\{7\}.header\textquoteright{} & ID:invalid\_data\_structure\tabularnewline
\hline 
7 & data\{4\} is a nx2 vector (instead of a nx1 vector) & ID:invalid\_data\_structure\tabularnewline
\hline 
8 & the length of data\{1\}.data is incompatible with the duration & ID:invalid\_data\_structure\tabularnewline
\hline 
9 & An entry of data\{2\}.data is larger than \textquoteleft duration\textquoteright{} & ID:invalid\_data\_structure\tabularnewline
\hline 
10 & data\{5\} has an non-existent chantype (\textquoteleft scanner\textquoteright ) & ID:invalid\_data\_structure\tabularnewline
\hline 
11 & duplicates (9) with struct chan input & ID:invalid\_data\_structure\tabularnewline
\hline 
\end{tabular}

\subsubsection*{Return all channels }

Function name: valid\_datafile\_0(testCase) \\
Description: Checks the function, if all channels shall be returned
(chan = 0). 

\subsubsection*{Return all channels (struct input) }

Function name: valid\_datafile\_1(testCase) \\
Description: Checks the function, if all channels shall be returned
(chan = 0) and the input is a struct. 

\subsubsection*{Return one channel}

Function name: valid\_datafile\_2(testCase) \\
Description: Checks the function, if only one channel shall be returned
(chan = 2). 

\subsubsection*{Return one channel}

Function name: valid\_datafile\_3(testCase) \\
Description: Checks the function, if multiple channels shall be returned
(chan = {[}3 5{]}). 

\subsubsection*{Return scr channels}

Function name: valid\_datafile\_4(testCase) \\
Description: Checks the function, if only the scr channels shall be
returned. 

\subsubsection*{Return event channels}

Function name: valid\_datafile\_5(testCase) \\
Description: Checks the function, if only the event channels shall
be returned. 

\subsubsection*{Save data }

Function name: valid\_datafile\_6(testCase) \\
Description: Checks the function, if data is to be saved (chan struct). 

\subsection{Testcases: pspm\_pp}

\subsubsection{Information}

Testclass: pspm\_pp\_test \\
Function: newfile = pspm\_pp('median', datafile, n, channelnumber)
or newfile = pspm\_pp('butter', datafile, freq, channelnumber) 

\subsubsection{Testcases}

\subsubsection*{Invalid input}

Function name: invalid\_input(this) \\
Description: Checks for warnings, if the input arguments are invalid.
\\
\\
Tests: 

\begin{tabular}{|c|c|}
\hline 
Input & Expected warning\tabularnewline
\hline 
\hline 
pspm\_pp('butter', 'file') {[}no freq{]} & ID:invalid\_input \tabularnewline
\hline 
pspm\_pp('foo', 'file', 100) {[}no valid first argument{]} & ID:invalid\_input \tabularnewline
\hline 
pspm\_pp('butter', 'file', 19) {[}freq below 20{]} & ID:invalid\_input \tabularnewline
\hline 
\end{tabular}

\subsubsection*{Median test}

Function name: median\_test(this) \\
Description: Checks medianfilter functionality \\
\\
Setup: \\
\\
Testfile with 3 channels (scr, hb, scr).\\
\\
Tests:
\begin{enumerate}
\item Filter one channel {[}Input: newfile = pspm\_pp('median', testfile,
50, 3){]} 

i. Check if sts == 1, when data is loaded with pspm\_load\_data. 

ii. Check if newfile has the same number of channels as testfile
\item Filter multiple channel {[}Input: newfile = pspm\_pp('median', testfile,
50){]} 

i. Check if sts == 1, when data is loaded with pspm\_load\_data. 

ii. Check if newfile has the same number of channels as testfile 
\end{enumerate}

\subsubsection*{Butterworth filter test}

Function name: butter\_test(this) \\
Description: Checks Butterworth filter functionality \\
\\
Setup: \\
\\
Testfile with 3 channels (scr, hb, scr). \\
\\
Tests:
\begin{enumerate}
\item Filter one channel {[}Input: newfile = pspm\_pp('butter', testfile,
40, 3){]} 

i. Check if sts == 1, when data is loaded with pspm\_load\_data. 

ii. Check if newfile has the same number of channels as testfile
\item Filter multiple channel {[}Input: newfile = pspm\_pp('butter', testfile,
40){]} 

i. Check if sts == 1, when data is loaded with pspm\_load\_data. 

ii. Check if newfile has the same number of channels as testfile 
\end{enumerate}

\subsection{Testcases: pspm\_prepdata}

\subsubsection{Information}

Testclass: pspm\_prepdata\_test \\
Function: {[}sts, outdata, newsr{]} = pspm\_prepdata(data, filt) 

\subsubsection{Testcases}

\subsubsection*{Invalid input}

Function name: invalid\_input(this) \\
Description: Checks for warnings, if the input arguments are invalid.
\\
\\
Tests: 

\begin{tabular}{|c|c|}
\hline 
Input & Expected warning\tabularnewline
\hline 
\hline 
pspm\_prepdata({[}1 NaN 3{]}, filt) {[}NaN values in data{]} & ID:invalid\_input\tabularnewline
\hline 
pspm\_prepdata({[}1 2 3{]}) {[}no filt variable{]} & ID:invalid\_input \tabularnewline
\hline 
pspm\_prepdata(data, filt) {[}filt has no hporder field{]} & ID:invalid\_input \tabularnewline
\hline 
pspm\_prepdata('foo', filt) {[}no numeric data{]} & ID:invalid\_input \tabularnewline
\hline 
pspm\_prepdata(data, filt) {[}with lpfreq = 'foo' (not valid){]} & ID:invalid\_input \tabularnewline
\hline 
\end{tabular}

\subsubsection*{Hipassfilter test}

Function name: hipassfilter\_test(this) \\
Description: Checks hipassfilter functionality (without downsampling)
\\
\\
Setup: \\
data = rand(1000,1);\\
\\
filt.sr = 100; \\
filt.lpfreq = 'none'; \\
filt.lporder = 1; \\
filt.hpfreq = 20; \\
filt.hporder = 1; \\
filt.down = 'none';\\
\\
Tests:
\begin{enumerate}
\item Unidirectional tests {[}filt.direction = \textquoteleft uni\textquoteright {]}

i. Check if sts == 1 

ii. Check if newsr == filt.sr 

iii. Check if outdata is empty 

iv. Check if length(outdata) == length(data) 
\item Unidirectional tests {[}filt.direction = \textquoteleft bi\textquoteright {]}

i. Check if sts == 1 

ii. Check if newsr == filt.sr 

iii. Check if outdata is empty 

iv. Check if length(outdata) == length(data) 
\end{enumerate}

\subsubsection*{Lowpassfilter test}

Function name: lowpassfilter\_test(this) \\
Description: Checks hipassfilter functionality (without downsampling)
\\
\\
Setup: \\
data = rand(1000,1);\\
filt.sr = 100; \\
filt.lpfreq = 40; \\
filt.lporder = 1; \\
filt.hpfreq = 'none'; \\
filt.hporder = 1; \\
filt.down = 'none';\\
\\
Tests: \\
\\
Same tests as in hipassfilter\_test. Additionally there is a check
for a warning if filt.lpfreq is higher (or equal) than the nyquist
frequency: 

\begin{tabular}{|c|c|}
\hline 
Input & Expected warning\tabularnewline
\hline 
\hline 
pspm\_prepdata(data, filt) {[}filt.sr = 100; filt.lpfreq = 60{]} & ID:no\_low\_pass\_filtering\tabularnewline
\hline 
\end{tabular}

\subsubsection*{Bandpassfilter test}

Function name: bandpassfilter\_test(this) \\
Description: Checks bandpassfilter functionality (without downsampling)
\\
\\
Setup: \\
data = rand(1000,1);\\
filt.sr = 200;\\
filt.lpfreq = 99; \\
filt.lporder = 1; \\
filt.hpfreq = 20; \\
filt.hporder = 1; \\
filt.down = 'none';\\
\\
Tests: Same tests as in hipassfilter\_test. 

\subsubsection*{Integer samplerate ratio downsampling test}

Function name: int\_sr\_ratio\_downsample\_test(this) \\
Description: Checks downsampling functionality, if the ratio between
filt.sr and filt.down is an integer. \\
\\
Setup: \\
\\
ratio = 2; \%ratio between filt.sr and filt.down\\
\\
filt.down = 100; \\
filt.sr = ratio \\
filt.down; filt.lpfreq = 40; \\
filt.lporder = 1; \\
filt.hpfreq = 'none'; \\
filt.hporder = 1; \\
filt.direction = 'uni';\\
\\
data = rand(filt.sr {*} 10,1);\\
\\
Tests:
\begin{enumerate}
\item Check if sts == 1
\item Check if newsr == filt.down
\item Check if outdata is empty
\item Check if ratio{*}length(outdata) == length(data) 
\end{enumerate}

\subsection{Testcases: pspm\_process\_illuminance}

\subsubsection{Information}

Testclass: pspm\_process\_illuminance\_test\\
Function: {[}sts, out{]} = pspm\_process\_illuminance(ldata, sr, options)

\subsubsection{Setup}

This test class is parameterized. The test data is generated by the
function itself and when needed, files will be written to datafile<variable\_nr>.mat.

\subsubsection*{Test parameters}

These are parameters which define what kind of data should be passed
to pspm\_process\_illuminance and which options should be set.\\
\begin{tabular}{|c|>{\raggedright}p{10cm}|}
\hline 
\textbf{bf\_dur} & Defines the duration of the basis function.\tabularnewline
\hline 
\textbf{bf\_offset} & Defines the offset of the basis function.\tabularnewline
\hline 
\textbf{dur} & Defines the duration of the generated dataset.\tabularnewline
\hline 
\textbf{sr} & Defines the samplerate of the generated dataset.\tabularnewline
\hline 
\textbf{n\_times} & Defines how many datasets should be generated.\tabularnewline
\hline 
\textbf{mode} & Defines the whether the dataset should be written to a file, kept
as inline variable or should be a mix of both. Can be either 'file',
'inline' or 'mixed'\tabularnewline
\hline 
\textbf{overwrite} & Defines whether existing files should be overwritten or not.\tabularnewline
\hline 
\end{tabular}

\subsubsection{Testcases}

\subsubsection*{Invalid input}

Function name: invalid\_input(this) \\
Description: Checks for warnings, if the input arguments are invalid.
\\
\\
Tests: \\
\\
\begin{tabular}{|c|>{\centering}p{6cm}|c|}
\hline 
Test No. & Input & Expected warning\tabularnewline
\hline 
\hline 
1 & pspm\_process\_illuminance() {[}no arguments{]} & ID:invalid\_input\tabularnewline
\hline 
2 & pspm\_process\_illuminance({[}{]}) {[}empty data{]} & ID:missing\_data\tabularnewline
\hline 
3 & pspm\_process\_illuminance(1:10) {[}missing samplerate{]} & ID:invalid\_input\tabularnewline
\hline 
4 & pspm\_process\_illuminance(1:10, 'a') {[}invalid ssamplerate{]} & ID:invalid\_input\tabularnewline
\hline 
5 & pspm\_process\_illuminance(\{1:10\}, 1) {[}cell, no cell{]} & ID:invalid\_input\tabularnewline
\hline 
6 & pspm\_process\_illuminance(1:10, \{1\}) {[}no cell, cell{]} & ID:invalid\_input\tabularnewline
\hline 
7 & pspm\_process\_illuminance(\{1:10, 10:10\}, \{1\}) {[}different sized
cells{]} & ID:invalid\_input\tabularnewline
\hline 
8 & pspm\_process\_illuminance(\{1:10, 'a'\},\{1,2\}) {[}invalid file{]} & ID:non\_existent\_file\tabularnewline
\hline 
9 & pspm\_process\_illuminance(\{1:10, 1:10\}, \{1, 'a'\}) {[}invalid
samplerate{]} & ID:invalid\_input\tabularnewline
\hline 
10 & pspm\_process\_illuminance(\{1:10\}, \{1\}, 'o') {[}wrong options{]} & ID:invalid\_input\tabularnewline
\hline 
11 & pspm\_process\_illuminance(\{1:10\}, \{1\}, opt){[}wrong transfer
settings{]} & ID:invalid\_input\tabularnewline
\hline 
12 & pspm\_process\_illuminance(\{1:10\}, \{1\}, opt){[}wrong duration{]} & ID:invalid\_input\tabularnewline
\hline 
13 & pspm\_process\_illuminance(\{1:10\}, \{1\}, opt){[}wrong offset{]} & ID:invalid\_input\tabularnewline
\hline 
14 & pspm\_process\_illuminance(\{1:10\}, \{1\}, opt){[}wrong outputfile{]} & ID:invalid\_input\tabularnewline
\hline 
15 & pspm\_process\_illuminance(\{1:10\}, \{1\}, opt){[}format of ldata
and opt.fn differs{]} & ID:invalid\_input\tabularnewline
\hline 
16 & pspm\_process\_illuminance(\{1:10\}, \{1\}, opt){[}opt.overwrite is
not boolean{]} & ID:invalid\_input\tabularnewline
\hline 
\end{tabular}

\subsubsection*{Test options}

Function name: test\_options(this, sr, dur, bf\_dur, bf\_offset) \\
Description: Tries out different combination options to process the
generated illuminance data.\\
\\
Tests: 
\begin{enumerate}
\item Generate data with sr and dur
\item Set options according to bf\_dur and bf\_offset
\item Set expected warning according to sr{*}dur and sr{*}bf\_dur
\begin{enumerate}
\item expect empty data if sr{*}dur < 1
\item expect invalid\_input if sr{*}bf\_dur < 1
\item otherwise expect no warning
\end{enumerate}
\item Test if issued warning equals expected warning
\item Test if sts equals expected value
\item Test if amount of data elements of input and output data is equal
\end{enumerate}

\subsubsection*{Test multi}

Function name: test\_multi(this, n\_times, mode) \\
Description: Generates n sets of illuminance data and passes it to
pspm\_process\_illuminance.\\
\\
Tests: 
\begin{enumerate}
\item Generate data with 10 (sr), 100 (dur), n\_times (amount), mode
\item Test if pspm\_process\_illuminance issues no warning
\item Test if sts is 1
\item For n\_times == 1, test if out has 10{*}100 data points
\item for n\_times \ \textasciitilde = 1, test if output has same size
as input
\end{enumerate}

\subsubsection*{Test overwrite}

Function name: test\_overwrite(this, overwrite) \\
Description: Generate illuminance file and test overwrite behaviour.\\
\\
Tests: 
\begin{enumerate}
\item Generate data with 10 (sr), 100 (dur), 1 (amount), 'file'
\item Test if pspm\_process\_illuminance issues no warning
\item Test if sts equals 1
\item Test if existing file was overwriten or not
\end{enumerate}

\subsubsection{Other methods}

\subsubsection*{Generate lx}

Has some of the Test parameters as parameter implemented and accordingly
generates the lx data. According to the calling arguments the output
is a cell of files and data vectors. All generated files will be stored
in the property 'datafiles'. They will be removed once all tests have
finished.

\subsubsection*{Cleanup}

Located in MethodTeardown and is called once the test class has finished
all tests. It then removes all the datafiles which can be found in
the property 'datafiles'.

\subsection{Testcases: pspm\_pulse\_convert}

\subsubsection{Information}

Testclass: pspm\_pulse\_convert\_test\\
Function: wavedata = pspm\_pulse\_convert(pulsedata, resamplingrate,
samplingrate)

\subsubsection{Testcases}

\subsubsection*{Invalid input}

Function name: invalid\_input(testCase)\\
Description: Pass invalid input arguments and test if the error message
is correct.\\
\\
Tests:

\begin{tabular}{|c|c|}
\hline 
Input & Expected warning\tabularnewline
\hline 
\hline 
pspm\_pulse\_convert() & ID:invalid\_input \tabularnewline
\hline 
pspm\_pulse\_convert(10\textasciicircum -3 {*} (1:10000)') & ID:invalid\_input \tabularnewline
\hline 
pspm\_pulse\_convert(10\textasciicircum -3 {*} (1:10000)', 10000) & ID:invalid\_input\tabularnewline
\hline 
\end{tabular}

\subsubsection*{Valid input}

Function name: valid\_input(testCase)\\
Description: Pass generated, valid data and test if function issues
no warning.\\
\\
Tests:
\begin{enumerate}
\item Test function without downsampling the data
\item Test function with downsampling the data
\end{enumerate}

\subsection{Testcases: pspm\_ren}

\subsubsection{Information}

Testclass: pspm\_ren\_test \\
Function: out\_newfilename = pspm\_ren(filename, newfilename) 

\subsubsection{Testcases}

\subsubsection*{Invalid input}

Function name: invalid\_input (this) \\
Description: Checks for warnings, if the input arguments are invalid.\\
\\
Tests: 

\begin{tabular}{|c|c|}
\hline 
Input & Expected warning\tabularnewline
\hline 
\hline 
pspm\_ren('fn') {[}no newfilename{]} & ID:invalid\_input \tabularnewline
\hline 
pspm\_ren(\{'fn1', 'fn2'\}, \{'rfn1', 'rfn2', 'rfn3'\}) {[}non same
size cell arrays{]} & ID:invalid\_input \tabularnewline
\hline 
\end{tabular}

\subsubsection*{Char Valid Input}

Function name: char\_valid\_input (this) \\
Description: Checks the function if the input variables are of type
char. It uses pspm\_load\_data to check the files. \\
\\
Tests:
\begin{enumerate}
\item Check if out\_newefilename = newfilename
\item Check if sts==1 (of pspm\_load\_data output)
\item Check if the field \textquoteleft infos.rendata\textquoteright{} exists
\item Check if the field \textquoteleft infos.newname\textquoteright{} exists
\item Check if the original file has been deleted 
\end{enumerate}

\subsubsection*{Cell Valid Input}

Function name: cell\_valid\_input (this) \\
Description: Checks the function if the input variables are of type
cell. It uses pspm\_load\_data to check the files. \\
\\
Tests: \\
\\
The inputs are two-element cell arrays. For both elements the same
tests as in the char\_valid\_input function are performed individually.

\subsection{Testcases: pspm\_resp\_pp}

\subsubsection{Information}

Testclass: pspm\_resp\_pp\_test\\
Function: sts = pspm\_resp\_pp(fn, sr, chan, options)\\


\subsubsection{Testcases}

\subsubsection*{Regression Test against Revision r660}

Function name: compare\_results\_to\_results\_obtained\_from\_r660\_version(this)\\
Description: In r660, there was a bug found in pspm\_resp\_pp that
caused it to crash with index out of bounds error on inputs containing
some edgecase. This test specifically checks whether the fixed version
returns the same results as the version before the bugfix on data
that didn't cause a crash.\\
\\
Tests:
\begin{enumerate}
\item Check if the returned channel types have the same name and ordering
\item Check if the returned data is the same
\end{enumerate}

\subsection{Testcases: pspm\_split\_sessions}

\subsubsection{Information}

Testclass: pspm\_split\_sessions\_test \\
Properties: expected\_number\_of\_files = 3; \\
Function: newdatafile = pspm\_split\_sessions(datafile, markerchannel,
options) 

\subsubsection{Setup}

For the tests a testdatafile with three channels is used (duration
is 100s). The markerchannel data is: \\
\\
data = {[}1 4 9 12 30 31 34 41 43 59 65 72 74 80 89 96{]}' \\
\\
Hence if MAXSN=10 \& BRK2NORM=3 (default values) the datafiles should
be split into 3 files. If different values are being used, update
the property \textquoteleft expected\_number\_of\_files\textquoteright{}
of the testclass object accordingly. 

\subsubsection{Testcases}

\subsubsection*{Invalid input}

Function name: invalid\_input (this) \\
Description: Checks for warnings, if the input arguments are invalid.\\
\\
Tests: 

\begin{tabular}{|c|c|}
\hline 
Input & Expected warning\tabularnewline
\hline 
\hline 
pspm\_split\_sessions() {[}no filename{]} & ID:invalid\_input \tabularnewline
\hline 
pspm\_split\_sessions (2) {[}no string filename{]} & ID:invalid\_input \tabularnewline
\hline 
pspm\_split\_sessions (\textquoteleft fn\textquoteright , \textquoteleft foo\textquoteright )
{[}no numeric marker channel no.{]} & ID:invalid\_input \tabularnewline
\hline 
\end{tabular}

\subsubsection*{One datafile}

Function name: one\_datafile(this) \\
Description: Checks the function if the variable \textquoteleft datafile\textquoteright{}
is of type char (one datafile). The markerchannel number is not assigned
explicitly.\\
\\
Tests:
\begin{enumerate}
\item Check if the file has been split into \textquoteleft expected\_number\_of\_files\textquoteright{}
files For each output file the following tests are performed:
\item Check if sts == 1, when data is loaded with pspm\_load\_data.
\item Check if number of channels is correct.
\item Check it the field infos.slitdate exists
\item Check if the field infos.splitsn exists
\item Check if the field infos.splitfile exists. 
\end{enumerate}

\subsubsection*{Multiple datafiles}

Function name: multiple\_datafiles(this) \\
Description: Checks the function if the variable \textquoteleft datafile\textquoteright{}
is of type cell (two datafiles). The markerchannel number is assigned
explicitly.\\
\\
Tests: \\
\\
For both datafiles the same tests as in the one\_datafile function
are performed individually. Additionally it is tested if the number
of input files does match the number of output files. 

\subsection{Testcases: pspm\_trim}

\subsubsection{Information}

Testclass: pspm\_trim\_test \\
Function: newdatafile=pspm\_trim(datafile, from, to, reference, options) 

\subsubsection{Setup}

If not otherwise declared, the input variable fn is referring to a
datafile which was generated with pspm\_testdata\_gen and consists
of the following channels: \\
\\
data\{1\}.chantype = 'scr'; \\
data\{2\}.chantype = 'marker'; \\
data\{3\}.chantype = 'hr'; \\
data\{4\}.chantype = 'hb'; \\
data\{5\}.chantype = 'marker'; \\
data\{6\}.chantype = 'resp'; \\
data\{7\}.chantype = 'scr';\\
\\
The duration of the data recording is 10s. 

\subsubsection{Testcases}

\subsubsection*{Invalid input arguments }

Function name: invalid\_inputargs(testCase) \\
Description: Checks for warnings, if the input arguments are invalid.
\\
\\
Tests: \\
\begin{tabular}{|>{\centering}p{6cm}|c|}
\hline 
Input & Expected warning\tabularnewline
\hline 
\hline 
pspm\_trim(testCase.fn, {[}1 2{]}, 5, 'marker') {[}invalid from parameter{]} & ID:invalid\_input\tabularnewline
\hline 
pspm\_trim(testCase.fn, 0, 'bla', 'marker') {[}invalid to parameter{]} & ID:invalid\_input\tabularnewline
\hline 
pspm\_trim(testCase.fn, 0, '{[}{]}', 'marker') {[}invalid to parameter{]} & ID:invalid\_input\tabularnewline
\hline 
pspm\_trim(fn, 0, 5) {[}no reference{]} & ID:invalid\_input\tabularnewline
\hline 
pspm\_trim(fn, 0, 5, 6) {[}no char or 2-element numeric reference{]} & ID:invalid\_input\tabularnewline
\hline 
pspm\_trim(fn, 0, 5, \textquoteleft bla\textquoteright ) {[}invalid
char reference{]} & ID:invalid\_input\tabularnewline
\hline 
pspm\_trim(fn, 0, 5, {[}-1 5{]}) {[}invalid numeric start reference{]} & ID:invalid\_input\tabularnewline
\hline 
pspm\_trim(fn, 0, 5, {[}5 4{]}) {[}invalid numeric start/end reference{]} & ID:invalid\_input\tabularnewline
\hline 
\end{tabular}

\subsubsection{Reference = \textquoteleft marker\textquoteright{} tests}

Function name: marker\_tests(testCase) \\
Description: A wrapper function for tests with reference = \textquoteleft marker\textquoteright .
It executes the methods markertest\_k, where the testcases are defined. 

\subsubsection*{markertest\_1}

Description: from and to are set so that the trimming points are out
of the range {[}0,duration{]}. Hence the data should not be trimmed.
\\
\\
Expected warning: ID: marker\_out\_of\_range \\
\\
Input: pspm\_trim(fn, -20, 20, \textquoteright marker\textquoteright ) 

\subsubsection*{markertest\_2 }

Description: from and to are set so that the trimming points are exactly
(0, duration). Hence the data should not be trimmed. \\
\\
Input: from = -1 {*} marker(1) to = duration - marker(end) pspm\_trim(fn,
from, to, \textquoteright marker\textquoteright ) 

\subsubsection*{markertest\_3 }

Description: from and to are set so that the trimming points in the
range {[}0,duration{]}. \\
\\
Input: pspm\_trim(fn, 1, -2, \textquoteright marker\textquoteright ) 

\subsubsection{Reference = \textquoteleft file\textquoteright{} tests}

Function name: file\_tests(testCase) \\
\\
Description: A wrapper function for tests with reference = \textquoteleft file\textquoteright .
It executes the methods filetest\_k, where the testcases are defined. 

\subsubsection*{filetest\_1}

Description: from and to are set so that the trimming points are out
of the range {[}0,duration{]}. Hence the data should not be trimmed.
\\
\\
Expected warning: ID: marker\_out\_of\_range \\
\\
Input: pspm\_trim(fn, -12.5, 50, \textquoteright marker\textquoteright ) 

\subsubsection*{filetest\_2}

Description: from and to are set so that the trimming points are exactly
(0, duration). Hence the data should not be trimmed. \\
\\
Input: pspm\_trim(fn, 0 , duration, \textquoteright marker\textquoteright ) 

\subsubsection*{filetest\_3}

Description: from and to are set so that the trimming points in the
range {[}0,duration{]}. \\
\\
Input: pspm\_trim(fn,2.1, duration -- 2.5, \textquoteright marker\textquoteright ) 

\subsubsection*{Numeric reference tests}

Function name: num\_tests(testCase) \\
Description: A wrapper function for tests with reference = {[}a b{]}
(a, b are two integers with a<b). It executes the methods markertest\_k,
where the testcases are defined. 

\subsubsection*{numtest\_1}

Description: from and to are set so that the trimming points are out
of the range {[}0,duration{]}. Hence the data should not be trimmed.
\\
\\
Expected warning: ID: marker\_out\_of\_range \\
\\
Input: pspm\_trim(fn, -20, 20, {[}2 14{]}) 

\subsubsection*{numtest\_2}

Description: from and to are set so that the trimming points are exactly
(0, duration). Hence the data should not be trimmed. \\
\\
Input: from = -1 {*} marker(3) to = duration - marker(8) pspm\_trim(fn,
from, to, {[}3 8{]}) 

\subsubsection*{numtest\_3}

Description: from and to are set so that the trimming points in the
range {[}0,duration{]}. \\
\\
Input: pspm\_trim(fn, -1.5, 2, {[}2 7{]}) 

\subsubsection*{numtest\_4}

Description: Second reference point is out of the marker range; from
is set to \textquoteleft none\textquoteright . Hence the data should
not be trimmed. \\
\\
Expected warning: ID: marker\_out\_of\_range \\
\\
Input: pspm\_trim(fn, \textquoteleft none\textquoteright , 0, {[}1
(numel(marker) + 1){]}) 

\subsubsection*{Multiple file reference tests}

Function name: multiple\_files(testCase) \\
\\
Description: The input variable datafile is either a cell array of
two filenames or a cell array of two stucts. In both cases it is tested
whether the return value is also a cell array of two filenames and
whether both files are trimmed correctly. 

\subsubsection*{Options tests}

\subsubsection*{Marker channel number option}

Function name: marker\_chan\_num\_option\_test(testCase) \\
\\
Description: Tests if the option marker\_chan\_num is working correctly.
There are two tests: Test 1: Checks for a warning if the selected
channel is no marker channel. Test 2: Checks if the selected channel
is actually used. 

\subsection{Testcases: pspm\_write\_channel}

\subsubsection{Information}

Testclass: pspm\_write\_channel\_test \\
Function: {[}sts{]} = pspm\_write\_channel(fn, newdata, action, options)

\subsubsection{Setup}

\subsubsection*{Testdatafile}

The testdatafile is a class property. It is generated by the function
generate\_testdatafile() once the test class is setup. Changes made
by a test to the testdatafile won't be reverted. Thus some test functions
rely on the changes made by another test function. Therefore the functions
may not work properly if called individually.

\paragraph*{Structure (created with generate\_testdatafile())}

\texttt{data\{1\}.chantype = 'scr'; }~\\
\texttt{data\{2\}.chantype = 'marker';}~\\
\texttt{data\{3\}.chantype = 'scr'; }~\\
\texttt{}~\\
The sampling rate is 100 Hz and the duration is 500s.

\subsubsection{Testcases}

\subsubsection*{Invalid input}

Function name: invalid\_input(this) \\
Description: Checks for warnings, if the input arguments are invalid.
\\
\\
Tests: \\
\begin{tabular}{|>{\centering}p{6cm}|c|}
\hline 
Input & Expected warning\tabularnewline
\hline 
\hline 
pspm\_write\_channel() {[}no parameter{]} & ID:invalid\_input\tabularnewline
\hline 
pspm\_write\_channel(1) {[}fn is a number{]} & ID:invalid\_input\tabularnewline
\hline 
pspm\_write\_channel('some\_file', {[}{]}) {[}no action passed{]} & ID:unknown\_action\tabularnewline
\hline 
pspm\_write\_channel('some\_file', {[}{]}, '') {[}empty action passed{]} & ID:unknown\_action\tabularnewline
\hline 
options.channel = 'some invalid channel'

pspm\_write\_channel('some\_file', {[}{]}, 'add', options) {[}invalid
channel{]} & ID:invalid\_input\tabularnewline
\hline 
options.channel = -1

pspm\_write\_channel('some\_file', {[}{]}, 'add', options) {[}negative
channel{]} & ID:invalid\_input\tabularnewline
\hline 
options.channel = 0

pspm\_write\_channel('some\_file', {[}{]}, 'delete', options) {[}no
channel and no data given{]} & ID:invalid\_input\tabularnewline
\hline 
options.channel = 0

pspm\_write\_channel('some\_file', {[}{]}, 'add', options) {[}empty
newdata{]} & ID:invalid\_input\tabularnewline
\hline 
options.channel = 0

pspm\_write\_channel('some\_file', 1:3, 'add', options) {[}newdata
is not cell and not struct{]} & ID:invalid\_input\tabularnewline
\hline 
options.channel = 1:5

pspm\_write\_channel(this.testdatafile, {[}{]}, 'delete', options)
{[}more given channels than in file exist{]} & ID:invalid\_input\tabularnewline
\hline 
options.channel = 'ecg'; pspm\_write\_channel(this.testdatafile, {[}{]},
'delete', options) & ID:no\_matching\_channels\tabularnewline
\hline 
pspm\_write\_channel(this.testdatafile, gen\_data.data\{1\}, 'add')
{[}generated data has the wrong format (two rows in one channel){]} & ID:invalid\_data\_structure\tabularnewline
\hline 
\end{tabular}

\subsubsection*{Action 'add'}

Function name: test\_add(this) \\
Description: Checks if action 'add' behaves as expected. A new channel
with chantype = 'hb', sr = 200 and duration = 500 is generated.\\
\\
Tests: 
\begin{enumerate}
\item Load condition before and after and pass it to 'Verify write'
\end{enumerate}

\subsubsection*{Action 'add transposed'}

Function name: test\_add\_transposed(this) \\
Description: Checks if action 'add' behaves as expected, when data
has the wrong dimensions. A new channel with chantype = 'rs', sr =
200 and duration = 500 is generated.\\
\\
Tests: 
\begin{enumerate}
\item Transpose generated data
\item Load condition before and after and pass it to 'Verify write'
\end{enumerate}

\subsubsection*{Action 'replace'/'add'}

Function name: test\_replace\_add(this) \\
Description: Checks if action 'replace' behaves as expected. A new
channel with chantype = 'hr', sr = 10 and duration = 500 is generated.\\
\\
Tests: 
\begin{enumerate}
\item Running pspm\_write\_channel with action = 'replace' should issue
'ID:no\_matching\_channels' (channeltype should not exist before)
and then instead add the channel
\item Load condition before and after and pass it to 'Verify write'
\end{enumerate}

\subsubsection*{Action 'replace'}

Function name: test\_replace(this) \\
Description: Checks if action 'replace' behaves as expected. A new
channel with chantype = 'hr', sr = 20 and duration = 500 is generated.\\
\\
Tests: 
\begin{enumerate}
\item Load condition before and after and pass it to 'Verify write'
\item Test if 'hr' channel has sample rate 20
\end{enumerate}

\subsubsection*{Action 'delete' (one channel)}

Function name: test\_delete\_single(this) \\
Description: Checks if action 'delete' behaves as expected. In this
test only one channel will be deleted. To test the delete algorithm
there will be 7 channels added which are then also used for test\_delete\_multi(this).
The particular channels are then identified by the sample rate which
corresponds to the channel id {*} 10.\\
\\
Tests:
\begin{enumerate}
\item Delete channel with chantype = 'hr' in newdata.header.chantype

\begin{enumerate}
\item Verify write
\item Ensure only one channel has been deleted
\item Test if there is no more channel with chantype = 'hr'
\end{enumerate}
\item Delete channel with channel number in options.channel

\begin{enumerate}
\item Verify Write
\item Ensure only one channel has been deleted
\end{enumerate}
\item Test the delete algorithm

\begin{enumerate}
\item Remove 'resp' channel with options.delete = 'last'

\begin{enumerate}
\item Verify write
\item Ensure only one channel has been deleted
\item Test if last channel was deleted 
\end{enumerate}
\item Remove 'resp' channel with options.delete = 'first'

\begin{enumerate}
\item Verify write
\item Ensure only one channel has been deleted
\item Test if last entry was not deleted
\end{enumerate}
\end{enumerate}
\end{enumerate}

\subsubsection*{Action 'delete' (multiple channels)}

Function name: test\_delete\_multi(this) \\
Description: Checks if action 'delete' behaves as expected. In this
test only multiple channels will be deleted. This test relys on the
changes made to the testdatafile by other test functions in this class.
\\
\\
Tests:
\begin{enumerate}
\item Delete channel 1 and 2 from testdatafile

\begin{enumerate}
\item Verify write
\item Ensure two channels have been deleted
\end{enumerate}
\item Delete all 'resp' channels from testdatafile

\begin{enumerate}
\item Verify write
\item Test if datafile contains no more 'resp' channels
\end{enumerate}
\end{enumerate}

\subsubsection{Other methods}

\subsubsection*{Verify write}

Is called after pspm\_write\_channel has been called (action = 'add'
or action = 'replace') and tests if data was written and a new history
entry was made. \\
Tests: 
\begin{enumerate}
\item if action = 'add', test if there is a new channel
\item if action = 'replace', test if there is still the same amount of channels
\item if action = 'delete', test if there have been as many channels deleted
as given in outinfos.channel
\item test if history has a new entry
\item search for channels with same chantype as added channel (should be
only one channel)
\item test if number of data elements in new channel and added channel is
equal
\item test if new channel and added channel have same 'sr'
\end{enumerate}

\section{External functions and tools}

\subsection{VB (Variational Bayes) inversion algorithm by Jean Daunizeau}

Updated October 2014

Changes made for use in PsPM: 
\begin{itemize}
\item \emph{VBA\_ReDisplay.m}, fixed try-catch syntax in various places
by adding a comma after ``try'' to avoid warning in matlab > 2007
\item \textit{VBA\_inv.m}, line 42: added warning off/on to suppress the
warning ``Matrix is singular, close to singular or badly scaled.
Results may be inaccurate. RCOND = NaN.''
\end{itemize}
Updated October 2016

Changes made for use in PsPM: 
\begin{itemize}
\item \emph{VBA\_ReDisplay.m}, fixed try-catch syntax in various places
by adding a comma after ``try'' to avoid warning in matlab > 2007
\item \textit{VBA\_inv.m}, line 48: added warning off/on to suppress the
warning ``Matrix is singular, close to singular or badly scaled.
Results may be inaccurate. RCOND = NaN.''
\item \textit{VBA\_NLStateSpaceModel.m}: added resetting warning to preceeding
state.
\end{itemize}

\section{List of functions}

\begin{longtable}[c]{|c|>{\centering}p{4cm}|c|c|}
\hline 
Name & Main author & Test exists & Test Doc\tabularnewline
\hline 
\hline 
f\_SCR & Dominik Bach \& Jean Daunizeau & - & -\tabularnewline
\hline 
f\_SF & Dominik Bach & - & -\tabularnewline
\hline 
g\_SCR & Dominik Bach & - & -\tabularnewline
\hline 
pspm & Dominik Bach & x & x\tabularnewline
\hline 
scr & Dominik Bach & x & x\tabularnewline
\hline 
pspm\_align\_channels & Dominik Bach & x & x\tabularnewline
\hline 
pspm\_axpos & Dominik Bach & - & -\tabularnewline
\hline 
pspm\_bf\_brf & Saurabh Khemka \& Dominik Bach & - & -\tabularnewline
\hline 
pspm\_bf\_data & Dominik Bach & x & x\tabularnewline
\hline 
pspm\_bf\_FIR & Dominik Bach & - & -\tabularnewline
\hline 
pspm\_bf\_Fourier & Dominik Bach & - & -\tabularnewline
\hline 
pspm\_bf\_hprf & Dominik Bach & - & -\tabularnewline
\hline 
pspm\_bf\_hprf\_e & Tobias Moser & - & -\tabularnewline
\hline 
pspm\_bf\_hprf\_fc & Tobias Moser & - & -\tabularnewline
\hline 
pspm\_bf\_hprf\_fc\_f & Tobias Moser & - & -\tabularnewline
\hline 
pspm\_bf\_lcrf\_gm & Tobias Moser & - & -\tabularnewline
\hline 
pspm\_bf\_ldrf\_gm & Tobias Moser & - & -\tabularnewline
\hline 
pspm\_bf\_ldrf\_gu & Tobias Moser & - & -\tabularnewline
\hline 
pspm\_bf\_psrf\_fc & Tobias Moser & - & -\tabularnewline
\hline 
pspm\_rarf\_e & Tobias Moser & - & -\tabularnewline
\hline 
pspm\_rarf\_fc & Tobias Moser & - & -\tabularnewline
\hline 
pspm\_rfrrf\_e & Tobias Moser & - & -\tabularnewline
\hline 
pspm\_rprf\_e & Tobias Moser & - & -\tabularnewline
\hline 
pspm\_bf\_scrf\_f & Dominik Bach & - & -\tabularnewline
\hline 
pspm\_bf\_scrf & Dominik Bach & - & -\tabularnewline
\hline 
pspm\_bf\_spsrf\_box & Laure Ciernik & - & -\tabularnewline
\hline 
pspm\_bf\_spsrf\_gamma & Laure Ciernik & - & -\tabularnewline
\hline 
pspm\_butter & Dominik Bach & X & X\tabularnewline
\hline 
pspm\_compute\_visual\_angle & Laure Ciernik & - & -\tabularnewline
\hline 
pspm\_con1 & Dominik Bach & - & -\tabularnewline
\hline 
pspm\_con2 & Dominik Bach & - & -\tabularnewline
\hline 
pspm\_contrast & Dominik Bach & - & -\tabularnewline
\hline 
pspm\_convert\_area2diameter & Tobias Moser & - & -\tabularnewline
\hline 
pspm\_convert\_au2mm & Tobias Moser & - & -\tabularnewline
\hline 
pspm\_convert\_illum2lum & Tobias Moser & - & -\tabularnewline
\hline 
pspm\_convert\_lux2cdm2 & Tobias Moser & - & -\tabularnewline
\hline 
pspm\_convert\_mm2visdeg & Tobias Moser & - & -\tabularnewline
\hline 
pspm\_convert\_pixel2unit & Laure Ciernik & - & -\tabularnewline
\hline 
pspm\_convert\_unit & Tobias Moser & x & x\tabularnewline
\hline 
pspm\_convert\_visangle2sps & Laure Ciernik & - & -\tabularnewline
\hline 
pspm\_data\_editor & Tobias Moser & - & -\tabularnewline
\hline 
pspm\_dcm\_inv & Dominik Bach & - & -\tabularnewline
\hline 
pspm\_dcm & Dominik Bach & - & -\tabularnewline
\hline 
pspm\_denoise\_spike & Dominik Bach & - & -\tabularnewline
\hline 
pspm\_display & Philipp C Paulus & - & -\tabularnewline
\hline 
pspm\_down & Dominik Bach & x & -\tabularnewline
\hline 
pspm\_downsample & Dominik Bach & - & -\tabularnewline
\hline 
pspm\_ecg2hb & Philipp C Paulus & x & x\tabularnewline
\hline 
pspm\_ecg2hb\_amri & Eshref Yozdemir & x & -\tabularnewline
\hline 
pspm\_ecg\_editor & Tobias Moser & - & -\tabularnewline
\hline 
pspm\_exp & Dominik Bach & x & -\tabularnewline
\hline 
pspm\_extract\_segments & Tobias Moser & x & x\tabularnewline
\hline 
pspm\_filtfilt & Dominik Bach & x & x\tabularnewline
\hline 
pspm\_find\_channel & Dominik Bach & x & x\tabularnewline
\hline 
pspm\_find\_data\_epochs & Tobias Moser & - & -\tabularnewline
\hline 
pspm\_find\_sounds & Samuel Gerster & x & x\tabularnewline
\hline 
pspm\_find\_valid\_fixations & Tobias Moser & x & x\tabularnewline
\hline 
pspm\_get\_acq\_bioread & Tobias Moser & x & x\tabularnewline
\hline 
pspm\_get\_acq & Dominik Bach & x & x\tabularnewline
\hline 
pspm\_get\_acqmat & Dominik Bach & x & x\tabularnewline
\hline 
pspm\_get\_biograph & Dominik Bach & x & x\tabularnewline
\hline 
pspm\_get\_biosemi & Dominik Bach & x & x\tabularnewline
\hline 
pspm\_get\_biotrace & Dominik Bach & x & x\tabularnewline
\hline 
pspm\_get\_blink\_l & Tobias Moser & - & -\tabularnewline
\hline 
pspm\_get\_blink\_r & Tobias Moser & - & -\tabularnewline
\hline 
pspm\_get\_brainvis & Dominik Bach & x & x\tabularnewline
\hline 
pspm\_get\_cell & Dominik Bach & - & -\tabularnewline
\hline 
pspm\_get\_cnt & Dominik Bach & - & -\tabularnewline
\hline 
pspm\_get\_custom & Tobias Moser & - & -\tabularnewline
\hline 
pspm\_get\_ecg & Dominik Bach & x & x\tabularnewline
\hline 
pspm\_get\_edf & Tobias Moser & x & x\tabularnewline
\hline 
pspm\_get\_events & Dominik Bach & x & x\tabularnewline
\hline 
pspm\_get\_eyelink & Christoph Korn, Tobias Moser & x & x\tabularnewline
\hline 
pspm\_get\_gaze\_x\_l & Tobias Moser & - & -\tabularnewline
\hline 
pspm\_get\_gaze\_y\_l & Tobias Moser & - & -\tabularnewline
\hline 
pspm\_get\_gaze\_x\_r & Tobias Moser & - & -\tabularnewline
\hline 
pspm\_get\_gaze\_y\_r & Tobias Moser & - & -\tabularnewline
\hline 
pspm\_get\_hb & Dominik Bach & x & x\tabularnewline
\hline 
pspm\_get\_hp & Dominik Bach & - & -\tabularnewline
\hline 
pspm\_get\_hr & Dominik Bach & x & x\tabularnewline
\hline 
pspm\_get\_labchartmat\_ext & Dominik Bach & x & x\tabularnewline
\hline 
pspm\_get\_labchartmat\_in & Dominik Bach & x & x\tabularnewline
\hline 
pspm\_get\_marker & Dominik Bach & x & x\tabularnewline
\hline 
pspm\_get\_markerinfo & Dominik Bach & - & -\tabularnewline
\hline 
pspm\_get\_mat & Dominik Bach & x & x\tabularnewline
\hline 
pspm\_get\_obs & Linus Rüttimann & x & x\tabularnewline
\hline 
pspm\_get\_physlog & Tobias Moser & - & -\tabularnewline
\hline 
pspm\_get\_pupil & Dominik Bach & x & x\tabularnewline
\hline 
pspm\_get\_pupil\_l & Tobias Moser & - & -\tabularnewline
\hline 
pspm\_get\_pupil\_r & Tobias Moser & - & -\tabularnewline
\hline 
pspm\_get\_resp & Dominik Bach & x & x\tabularnewline
\hline 
pspm\_get\_rf & Dominik Bach & - & -\tabularnewline
\hline 
pspm\_get\_saccade\_l & Laure Ciernik & - & -\tabularnewline
\hline 
pspm\_get\_saccade\_r & Laure Ciernik & - & -\tabularnewline
\hline 
pspm\_get\_scr & Dominik Bach & x & x\tabularnewline
\hline 
pspm\_get\_smi & Eshref Yozdemir & - & -\tabularnewline
\hline 
pspm\_get\_sps & Laure Ciernik & - & -\tabularnewline
\hline 
pspm\_get\_spike & Dominik Bach & x & x\tabularnewline
\hline 
pspm\_get\_sound & Tobias Moser & - & -\tabularnewline
\hline 
pspm\_get\_timing & Dominik Bach & x & x\tabularnewline
\hline 
pspm\_get\_txt & Dominik Bach & x & x\tabularnewline
\hline 
pspm\_get\_vario & Dominik Bach & x & x\tabularnewline
\hline 
pspm\_get\_viewpoint & Eshref Yozdemir & - & -\tabularnewline
\hline 
pspm\_get\_wdq & Dominik Bach & - & -\tabularnewline
\hline 
pspm\_get\_wdq\_n & Tobias Moser & x & x\tabularnewline
\hline 
pspm\_glm\_recon & Dominik Bach & - & -\tabularnewline
\hline 
pspm\_glm & Dominik Bach & x & x\tabularnewline
\hline 
pspm\_hb2hp & Dominik Bach & x & x\tabularnewline
\hline 
pspm\_hb2hr & Dominik Bach & - & -\tabularnewline
\hline 
pspm\_import & Dominik Bach & x & x\tabularnewline
\hline 
pspm\_init & Dominik Bach & - & -\tabularnewline
\hline 
pspm\_interpolate & Tobias Moser & x & x\tabularnewline
\hline 
pspm\_jobman & Gabriel Gräni & - & -\tabularnewline
\hline 
pspm\_job\_create & Dominik Bach & - & -\tabularnewline
\hline 
pspm\_load\_data & Dominik Bach & x & x\tabularnewline
\hline 
pspm\_load1 & Dominik Bach & x & x\tabularnewline
\hline 
pspm\_load\_single\_chan & Eshref Yozdemir & - & -\tabularnewline
\hline 
pspm\_merge & Dominik Bach & - & -\tabularnewline
\hline 
pspm\_path & Eshref Yozdemir & x & -\tabularnewline
\hline 
pspm\_peakscore & Dominik Bach & - & -\tabularnewline
\hline 
pspm\_pp & Dominik Bach & x & x\tabularnewline
\hline 
pspm\_ppu2hb & Samuel Gerster & - & -\tabularnewline
\hline 
pspm\_predval & Dominik Bach & - & -\tabularnewline
\hline 
pspm\_prepdata & Dominik Bach & x & x\tabularnewline
\hline 
pspm\_process\_illuminance & Tobias Moser & x & -\tabularnewline
\hline 
pspm\_pulse\_convert & Dominik Bach & x & -\tabularnewline
\hline 
pspm\_pupil\_correct\_eyelink & Eshref Yozdemir & x & -\tabularnewline
\hline 
pspm\_pupil\_correct & Eshref Yozdemir & x & -\tabularnewline
\hline 
pspm\_pupil\_pp & Eshref Yozdemir & x & -\tabularnewline
\hline 
pspm\_pupil\_pp\_options & Eshref Yozdemir & - & -\tabularnewline
\hline 
pspm\_quit & Dominik Bach & - & -\tabularnewline
\hline 
pspm\_ren & Dominik Bach & x & x\tabularnewline
\hline 
pspm\_resp\_pp & Dominik Bach & - & -\tabularnewline
\hline 
pspm\_rev\_con & Dominik Bach & - & -\tabularnewline
\hline 
pspm\_rev\_dcm & Dominik Bach & - & -\tabularnewline
\hline 
pspm\_rev\_glm & Dominik Bach & - & -\tabularnewline
\hline 
pspm\_rev2 & Dominik Bach & - & -\tabularnewline
\hline 
pspm\_review & Gabriel Graeni & - & -\tabularnewline
\hline 
pspm\_segment\_mean & Tobias Moser & - & -\tabularnewline
\hline 
pspm\_scr\_pp & Dadi Zhao & x & x\tabularnewline
\hline 
pspm\_sf\_auc & Dominik Bach & - & -\tabularnewline
\hline 
pspm\_sf\_dcm & Dominik Bach & - & -\tabularnewline
\hline 
pspm\_sf\_mp & Dominik Bach & - & -\tabularnewline
\hline 
pspm\_sf\_scl & Dominik Bach & - & -\tabularnewline
\hline 
pspm\_sf\_theta & Dominik Bach & - & -\tabularnewline
\hline 
pspm\_sf & Dominik Bach & - & -\tabularnewline
\hline 
pspm\_sf\_get\_theta & Dominik Bach & - & -\tabularnewline
\hline 
pspm\_show\_arms & Dominik Bach & - & -\tabularnewline
\hline 
pspm\_spike\_convert & Dominik Bach & - & -\tabularnewline
\hline 
pspm\_split\_sessions & Linus Rüttimann & x & x\tabularnewline
\hline 
pspm\_transfer\_function & Dominik Bach & - & -\tabularnewline
\hline
pspm\_time2index & Dadi Zhao & x & x\tabularnewline
\hline 
pspm\_trim & Dominik Bach & x & x\tabularnewline
\hline 
pspm\_ui & Dadi Zhao & x & x\tabularnewline
\hline 
pspm\_version & Tobias Moser & - & -\tabularnewline
\hline 
pspm\_write\_channel & Tobias Moser & x & x\tabularnewline
\hline 
set\_blinks\_saccades\_to\_nan & Eshref Yozdemir & x & -\tabularnewline
\hline 
\end{longtable}
\end{document}
